\documentclass[authoryear,a4paper, 14pt]{scrartcl}
% Agregué tipo de papel (creo que defaultea a carta, pero no estoy seguro) y tamaño de letra para que no sea tan chica. Cambié el documentclass por uno que a mí me gusta más pero es una cuestión de gustos.

% Con esto agrandás los márgenes
\topmargin=-1.5cm
\textwidth=16cm
\textheight=24cm
\oddsidemargin=0cm

% Este es el paquete que hace que las "References se te ponga en castellano".
\usepackage[spanish]{babel}

\usepackage[T1]{fontenc}
\usepackage[utf8x]{inputenc}

% Podés cambiar lmodern por times. Es una cuestión de gustos.
\usepackage{times}

% apacite te deja usar el bibliographystyle apacite (fijate más abajo) que te pone las ref. bibliográficas en castellano (e.g., "y" en vez de "and").
% \usepackage{natbib}
\usepackage[natbibapa]{apacite}

% El paquete sectsty te permite usar los comandos de abajo para cambiar el tamaño de los encabezados. Yo lo uso porque no me gusta que sean tan grandes.
\usepackage{sectsty}
\sectionfont{\fontsize{14}{15}\selectfont}
\subsectionfont{\fontsize{13}{15}\selectfont}
\subsubsectionfont{\fontsize{13}{15}\selectfont}

% Con fancyhdr podés agregar un encabezado fancy. Hay comandos para editar el formato, pero fijate en internet. Entre las llaves de \rhead podés escribir algo que quieras que se alinee a la derecha en el encabezado (como si fueran dos columnas), pero cuidado porque se te puede superponer con el encabezado default. Es importante que fijes títulos cortos para tus secciones porque si no te van a quedar dos líneas de encabezado lo hacés así: \section[título_corto]{título_normal}
\usepackage{fancyhdr}
\rhead{}
\pagestyle{fancy}

% Lo malo de scrartcl es que te agranda mucho la letra del título y eso. Pero con estos comandos lo achicás.
\usepackage{etoolbox}
\makeatletter
\patchcmd{\@maketitle}{\huge}{\Large}{}{}
\makeatother
\setkomafont{author}{\large}
\setkomafont{date}{\large}

\usepackage[margin=1in]{geometry}
\usepackage{hyperref}
\usepackage{longtable,booktabs}
\usepackage{graphicx,grffile}
\usepackage[normalem]{ulem}

\title{Diseño y lógica de los estudios cuantitativos}
\author{Santiago Gualchi - Federico Alvarez}
\date{21 de septiembre de 2019}

%\bibliographystyle{apacite}

\begin{document}
    \pagenumbering{gobble}
    \maketitle
    \pagenumbering{arabic}


\hypertarget{introduccion}{%
\section{Introducción}\label{introduccion}}

En la
\href{https://drive.google.com/file/d/1vRmq8vyfwkgANQznM2oPuzhorcoIaSkZ/view?usp=sharing}{clase
pasada}, realizamos una aproximación a la estadística y discutimos su
relevancia a la hora de estudiar el lenguaje. Establecimos que la estadística es la disciplina que se ocupa
de todas las etapas que involucran a los datos, incluyendo el diseño
previo a su recolección, su análisis e interpretación, y su
comunicación. En esta línea, introdujimos también una serie de
conceptos, entre ellos: población, muestra, variable, hipótesis, espacio
muestral y probabilidad.

Asimismo, establecimos la diferencia entre estadística descriptiva y
estadística inferencial. La primera se refiere al conjunto de técnicas
matemáticas que se limitan a describir las propiedades de la muestra
estudiada. La segunda alude a las pruebas que permiten generalizar las
observaciones sobre la muestra a la población relevante.

En esta reunión, vamos a avanzar sobre las líneas propuestas en el
encuentro anterior. Nos vamos a concentrar en las etapas de diseño de
una investigación para lo cual vamos a profundizar algunos de los
conceptos que ya introdujimos. Vamos a centrarnos en las hipótesis y
variables, y a estudiar sus propiedades y las repercusiones que traen
las distintas formas de operacionalizarlas. Vamos a explicar cómo
recolectar los datos de forma rigurosa y ver algunos buenos hábitos
para su almacenamiento.

\hypertarget{ejercitacion}{%
\subsection{Ejercitación}\label{ejercitacion}}

\begin{enumerate}
    \item
      Antes de avanzar, escribí definiciones para los siguientes conceptos:
      hipótesis, variable y operacionalización. No te preocupes si algunas
      de estas nociones te resultan muy nuevas. Está bien si las definís
      como te salga.
    \item
      ¿Cómo caracterizarías los procesos de recolección y almacenamiento de
      datos? ¿Qué cuidados tendrías a la hora de llevarlos a cabo?
    \item
      ¿Cómo pensás que podemos hacer para saber si nuestra hipótesis es
      correcta o incorrecta?
\end{enumerate}



\hypertarget{scouting}{%
\section{\texorpdfstring{\emph{Scouting}}{Scouting}}\label{scouting}}

Al principio de una investigación se suelen llevar a cabo las siguientes
tareas:

\begin{itemize}
    \item
      una primera caracterización del fenómeno;
    \item
      estudio de la bibliografía relevante;
    \item
      observación del fenómeno en escenarios naturales para posibilitar una
      primera generalización inductiva;
    \item
      recolección de información adicional (e.g., de colegas, estudiantes,
      etc.);
    \item
      razonamiento deductivo.
\end{itemize}

Si estudiamos el orden de palabras de los verbos frasales del inglés,
encontramos la siguiente alternancia:

\begin{enumerate}
\def\labelenumi{(\arabic{enumi})}
\item
  \begin{enumerate}
  \def\labelenumii{\alph{enumii}.}
  \item
    He picked up {[}\textsubscript{SN} the book{]}.

    Orden: \emph{VPO} (verbo - partícula - objeto)
  \item
    He picked {[}\textsubscript{SN} the book{]} up.

    Orden: \emph{VOP} (verbo - objeto- partícula)
  \end{enumerate}
\end{enumerate}

Al observar este fenómeno podemos encontrar un gran número de posibles
variables que podrían influir en la elección de una u otra forma. Las
\textbf{variables} son símbolos que pueden tomar, por lo menos, dos
estados o niveles diferentes (e.g., la edad de un grupo de estudiantes
de secundaria). En este sentido, se oponen a las \textbf{constantes},
que siempre presentan un mismo valor sin experimentar variación (e.g.,
la edad de un grupo de jóvenes de 12 años). Entre las variables que
pueden afectar al posicionamiento de la partícula en los verbos frasales
del inglés, las siguientes han sido propuestas en la bibliografía:

\begin{itemize}
    \item
      Complejidad del OD (Fraser, 1966);
    \item
      Largo del OD (Chen, 1986; Hawkins, 1994);
    \item
      Presencia de un SP direccional (Chen, 1986);
    \item
      Animacidad (Gries, 2003);
    \item
      Concreción (Gries, 2003); y
    \item
      Tipo del OD (Van Dongen, 1919), entre otras.
\end{itemize}

Esta información puede ser más fácilmente visualizada en formato
tabular, que permite reconocer qué variables han sido consideradas en
los distintos estudios y cuántas variables consideró cada estudio (véase
Cuadro 1).

\begin{longtable}[]{@{}lccccc@{}}
\caption{\protect\hypertarget{cuadro1}{}{Resumen} de la bibliografía
sobre posicionamiento de partículas en inglés I.}\tabularnewline
\toprule
\begin{minipage}[b]{0.18\columnwidth}\raggedright
\strut
\end{minipage} & \begin{minipage}[b]{0.15\columnwidth}\centering
Van Dongen (1919)\strut
\end{minipage} & \begin{minipage}[b]{0.13\columnwidth}\centering
Fraser (1966)\strut
\end{minipage} & \begin{minipage}[b]{0.13\columnwidth}\centering
Chen (1986)\strut
\end{minipage} & \begin{minipage}[b]{0.13\columnwidth}\centering
Hawkins (1994)\strut
\end{minipage} & \begin{minipage}[b]{0.13\columnwidth}\centering
Gries (2003)\strut
\end{minipage}\tabularnewline
\midrule
\endfirsthead
\toprule
\begin{minipage}[b]{0.18\columnwidth}\raggedright
\strut
\end{minipage} & \begin{minipage}[b]{0.15\columnwidth}\centering
Van Dongen (1919)\strut
\end{minipage} & \begin{minipage}[b]{0.13\columnwidth}\centering
Fraser (1966)\strut
\end{minipage} & \begin{minipage}[b]{0.13\columnwidth}\centering
Chen (1986)\strut
\end{minipage} & \begin{minipage}[b]{0.13\columnwidth}\centering
Hawkins (1994)\strut
\end{minipage} & \begin{minipage}[b]{0.13\columnwidth}\centering
Gries (2003)\strut
\end{minipage}\tabularnewline
\midrule
\endhead
\begin{minipage}[t]{0.18\columnwidth}\raggedright
Complejidad\strut
\end{minipage} & \begin{minipage}[t]{0.15\columnwidth}\centering
\strut
\end{minipage} & \begin{minipage}[t]{0.13\columnwidth}\centering
\(\times\)\strut
\end{minipage} & \begin{minipage}[t]{0.13\columnwidth}\centering
\strut
\end{minipage} & \begin{minipage}[t]{0.13\columnwidth}\centering
\strut
\end{minipage} & \begin{minipage}[t]{0.13\columnwidth}\centering
\strut
\end{minipage}\tabularnewline
\begin{minipage}[t]{0.18\columnwidth}\raggedright
Largo\strut
\end{minipage} & \begin{minipage}[t]{0.15\columnwidth}\centering
\strut
\end{minipage} & \begin{minipage}[t]{0.13\columnwidth}\centering
\strut
\end{minipage} & \begin{minipage}[t]{0.13\columnwidth}\centering
\(\times\)\strut
\end{minipage} & \begin{minipage}[t]{0.13\columnwidth}\centering
\(\times\)\strut
\end{minipage} & \begin{minipage}[t]{0.13\columnwidth}\centering
\strut
\end{minipage}\tabularnewline
\begin{minipage}[t]{0.18\columnwidth}\raggedright
SP Direccional\strut
\end{minipage} & \begin{minipage}[t]{0.15\columnwidth}\centering
\strut
\end{minipage} & \begin{minipage}[t]{0.13\columnwidth}\centering
\strut
\end{minipage} & \begin{minipage}[t]{0.13\columnwidth}\centering
\(\times\)\strut
\end{minipage} & \begin{minipage}[t]{0.13\columnwidth}\centering
\strut
\end{minipage} & \begin{minipage}[t]{0.13\columnwidth}\centering
\strut
\end{minipage}\tabularnewline
\begin{minipage}[t]{0.18\columnwidth}\raggedright
Animacidad\strut
\end{minipage} & \begin{minipage}[t]{0.15\columnwidth}\centering
\strut
\end{minipage} & \begin{minipage}[t]{0.13\columnwidth}\centering
\strut
\end{minipage} & \begin{minipage}[t]{0.13\columnwidth}\centering
\strut
\end{minipage} & \begin{minipage}[t]{0.13\columnwidth}\centering
\strut
\end{minipage} & \begin{minipage}[t]{0.13\columnwidth}\centering
\(\times\)\strut
\end{minipage}\tabularnewline
\begin{minipage}[t]{0.18\columnwidth}\raggedright
Concreción\strut
\end{minipage} & \begin{minipage}[t]{0.15\columnwidth}\centering
\strut
\end{minipage} & \begin{minipage}[t]{0.13\columnwidth}\centering
\strut
\end{minipage} & \begin{minipage}[t]{0.13\columnwidth}\centering
\strut
\end{minipage} & \begin{minipage}[t]{0.13\columnwidth}\centering
\strut
\end{minipage} & \begin{minipage}[t]{0.13\columnwidth}\centering
\(\times\)\strut
\end{minipage}\tabularnewline
\begin{minipage}[t]{0.18\columnwidth}\raggedright
Tipo\strut
\end{minipage} & \begin{minipage}[t]{0.15\columnwidth}\centering
\(\times\)\strut
\end{minipage} & \begin{minipage}[t]{0.13\columnwidth}\centering
\strut
\end{minipage} & \begin{minipage}[t]{0.13\columnwidth}\centering
\strut
\end{minipage} & \begin{minipage}[t]{0.13\columnwidth}\centering
\strut
\end{minipage} & \begin{minipage}[t]{0.13\columnwidth}\centering
\strut
\end{minipage}\tabularnewline
\bottomrule
\end{longtable}


\hypertarget{ejercitacion-3}{%
\subsection{Ejercitación}\label{ejercitacion-3}}

\begin{enumerate}
\def\labelenumi{\arabic{enumi}.}
\item
  Planteá un problema de investigación de tu interés.
\item
  ¿Cómo lo caracterizarías?
\end{enumerate}

\hypertarget{hipotesis-y-operacionalizacion}{%
\section{Hipótesis y
operacionalización}\label{hipotesis-y-operacionalizacion}}

Una vez que tenemos una visión general del fenómeno que queremos
estudiar, es el momento de formular \textbf{hipótesis}.

\hypertarget{hipotesis-cientificas-en-forma-de-texto}{%
\subsection{Hipótesis científicas en forma de
texto}\label{hipotesis-cientificas-en-forma-de-texto}}

Las hipótesis son:

\begin{itemize}
    \item
      enunciados generales ocupados de más de un evento singular;
    \item
      enunciados con una estructura condicional (\emph{si\ldots,
      entonces\ldots{}}), o que, al menos, puede ser parafraseados como tal;
      y
    \item
      potencialmente \textbf{falsables} (i.e., se pueden pensar eventos o
      situaciones que contradigan al enunciado) y \textbf{testeables} (i.e.,
      se pueden realizar pruebas que determinen la verdad o falsedad del
      enunciado).
\end{itemize}

Para el estudio del posicionamiento de partículas en inglés, podemos
pensar, por ejemplo, las siguientes hipótesis:

\begin{itemize}
    \item
      Si el objeto directo de un verbo frasal
      transitivo es sintácticamente complejo, entonces los hablantes nativos
      producirán el orden de constituyentes \emph{VPO} más frecuentemente que
      cuando el objeto directo es sintácticamente simple;
    \item
      Si el objeto directo de un verbo frasal transitivo es largo, entonces
      los hablantes nativos producirán el orden de constituyentes \emph{VPO}
      más seguido que cuando el objeto directo es corto; o
    \item
      Si una construcción de verbo-partícula es seguida por un SP
      direccional, entonces los hablantes nativos producirán el orden de
      constituyentes \emph{VOP} más seguido que cuando el SP direccional no
      está presente.
\end{itemize}

A su vez, las variables que consideremos pueden clasificarse según su
influencia:

\begin{description}
    \item[\label=\small{Variable independiente:}]
    Es la variable presente en la prótasis, y suele referirse a la causa de
    los cambios/efectos. La variable independiente representa tratamientos o
    condiciones que el investigador controla (directa o indirectamente) para
    entender sus efectos sobre la variable dependiente.
    \item[\label=\small{Variable dependiente:}]
    Es la variable presente en la apódosis, cuyos valores, variación o
    distribución se quieren explicar. La variable dependiente es la salida
    que depende del tratamiento experimental o de lo que el investigador
    cambia o manipula.
    \item[\label=\small{Variables de confusión o \emph{confounders}:}]
    Son variables que interactúan tanto con la variable independiente como
    con la variable dependiente. Es importante identificar los confounders
    para realizar mejores diseños experimentales y obtener resultados con
    menos ruido.
\end{description}

Una vez que formulamos nuestra hipótesis, a la que vamos a llamar
\textbf{hipótesis} \textbf{alternativa} (H\textsubscript{1}), y antes de
recolectar datos, tenemos que definir las condiciones
que van a falsar nuestra hipótesis. De este modo, definimos la
\textbf{hipótesis nula} (H\textsubscript{0}) como el opuesto lógico de
H\textsubscript{1} (predice la ausencia del efecto que enuncia
H\textsubscript{1}). La llamamos hipótesis nula porque se postula para
ser anulada con los datos de la investigación. Esto es importante,
porque la idea es que ambas hipótesis cubran todo el espacio de
resultados o \textbf{espacio muestral}, i.e., el conjunto de todos los
resultados teóricamente posibles. Por ejemplo:

\begin{itemize}
    \item
      Si el objeto directo de un verbo frasal transitivo es sintácticamente
      complejo, entonces los hablantes nativos \emph{no} producirán el orden
      de constituyentes VPO más seguido que cuando el objeto directo es
      sintácticamente simple (H\textsubscript{0} correspondiente a la
      primera hipótesis de tipo 1); o
    \item
      Los dos niveles de Orden (VPO y VOP) \sout{\emph{no} no} son
      igualmente frecuentes (H\textsubscript{0} correspondiente a la
      hipótesis de tipo 2).
\end{itemize}

Ahora bien, en algunas investigaciones es posible suponer que los
efectos o relaciones entre variables ocurran en una dirección
determinada (se desvíen de la H\textsubscript{0} hacia \emph{un} lado).
En estos casos, se dice que se establece una \textbf{hipótesis
direccional}. Por el contrario, las \textbf{hipótesis no direccionales}
solo predicen que existe un efecto o relación sin especificar la
dirección del efecto.

\hypertarget{operacionalizacion-de-variables}{%
\subsection{Operacionalización de
variables}\label{operacionalizacion-de-variables}}

Una vez que formulamos nuestra hipótesis, es importante encontrar un
modo de \textbf{operacionalizar} las variables. Esto supone decidir qué
será observado, contado, medido, etc. cuando investiguemos nuestras
hipótesis. Por ejemplo, si volvemos a las variables consideradas en la
bibliografía sobre el orden de palabras en los verbos frasales del
inglés, podemos operacionalizarlas como sigue:

\begin{itemize}
    \item
      Complejidad: OD \emph{simple} (e.g., \emph{the book}), \emph{OD
      modificado} \emph{sintagmáticamente} (e.g., \emph{the book on the
      table}) u \emph{OD modificado por} \emph{cláusula} (e.g., \emph{the
      book I had bought in Europe});
    \item
      Largo: el largo del OD medido en sílabas;
    \item
      SP direccional: \emph{presencia} o \emph{ausencia} de un SP
      direccional (e.g., \emph{He picked the book up {[}\textsubscript{SP}
      from the table{]}});
    \item
      Animacidad: \emph{animado} o \emph{inanimado};
    \item
      Concreción: \emph{concreto} o \emph{abstracto}; y
    \item
      Tipo del OD: \emph{pronominal} (e.g., \emph{He picked
      {[}\textsubscript{pron} him{]} up this morning}),
      \emph{semipronominal} (e.g., \emph{He picked {[}\textsubscript{semi}
      something{]} up from the floor}), \emph{léxico} (e.g., \emph{He picked
      {[}\textsubscript{léx} people{]} up this morning}) o \emph{nombre
      propio} (e.g., \emph{He picked {[}\textsubscript{prop} Peter{]} up
      this morning}).
\end{itemize}

Otro ejemplo, si queremos operacionalizar el conocimiento de una lengua
extranjera de una persona, podemos tomar en consideración:

\begin{itemize}
    \item
      La complejidad de las oraciones que una persona puede formar en la
      lengua en cuestión;
    \item
      El tiempo en segundos entre dos errores en la conversación;
    \item
      El número de errores cada 100 palabras en un texto que la persona
      escriba en 90 minutos.
\end{itemize}

La operacionalización de variables involucra el uso de \textbf{niveles}
numéricos para representar estados de variables. Un número puede ser una
medida (e.g., 402 ms de tiempo de reacción), pero estados discretos no 
numéricos también pueden, teóricamente, ser codificados usando números. 
Según los niveles de medida las variables pueden clasificarse en:

\begin{description}
    \item[\label=\small{Variable nominal (o binaria)}:]
    Sólo pueden tomar dos niveles diferentes y sus valores sólo revelan que
    los objetos con estos valores exhiben características diferentes (e.g.,
    animacidad);
    \item[\label=\small{Variable categórica:}]
    Son una generalización del caso anterior a tres niveles o más.
    (e.g., aspecto);
    \item[\label=\small{Variable ordinal:}]
    Distinguen categorías que presentan alguna clase de secuencia o progresión.
    \item[\label=\small{Variable cuantitativa:}]
    Consisten en valores numéricos que pueden ser continuos o discretos, para los cuales la diferencia absoluta entre los valores (variables de intervalo) o la proporción entre los valores (variable de razón) son significativas (e.g., largo en sílabas).
\end{description}

\hypertarget{hipotesis-estadisticas-en-formato-estadisticomatematico}{%
\subsection{Hipótesis estadísticas en formato
estadístico/matemático}\label{hipotesis-estadisticas-en-formato-estadisticomatematico}}

Después de formular las hipótesis (H\textsubscript{0} y
H\textsubscript{1}) en forma de texto y definir cómo operacionalizar las
variables, es necesario formular dos \textbf{versiones estadísticas} de
las hipótesis. Esto significa expresar los resultados numéricos
esperados sobre la base de las hipótesis textuales. Dichos resultados
suelen involucrar una de las siguientes formas matemáticas:

\begin{itemize}
    \item Frecuencias
    \item Promedios
    \item Dispersiones
    \item Correlaciones
    \item Distribuciones
\end{itemize}

Este va a ser el formato que vamos a usar para evaluar la
\textbf{significancia} de nuestras hipótesis, y su
definición va a depender directamente de cómo operacionalizamos las
variables. Por ejemplo, si nuestra hipótesis involucra la variable largo
del OD, su forma estadística no va a ser la misma si la operacionalizamos
cuantitativamente como largo medido en número de sílabas o de forma
discreta como una variable categórica con niveles \emph{corto},
\emph{mediano} y \emph{largo.} En el primer caso, nuestras hipótesis
estadísticas van a poder referirse a la media del largo, mientras que
esto no es posible en el segundo caso. Tomando largo como una variable
categórica podríamos operacionalizar nuestras hipótesis, por ejemplo,
basándonos en conteos o frecuencias.

Retomemos la H\textsubscript{1} respecto de la
presencia/ausencia de un SP direccional: si una construcción de
verbo-partícula es seguida por un SP direccional, entonces los hablantes
nativos producirán el orden de constituyentes VOP más seguido que cuando
el SP direccional no está presente. Si formulamos nuestras hipótesis
matemáticamente, obtenemos los siguientes resultados:

\[H_{1~\textrm{direccional}}: n_{\textrm{SSPP dir. en VPO}} < n_{\textrm{SSPP dir. en VOP}}\]
\[H_{1~\textrm{no direccional}}: n_{\textrm{SSPP dir. en VPO}} \neq n_{\textrm{SSPP dir. en VOP}}\]
\[H_{0}: n_{\textrm{SSPP dir. en VPO}} =n_{\textrm{SSPP dir. en VOP}}\]

\hypertarget{ejercitacion-4}{%
\subsection{Ejercitación}\label{ejercitacion-4}}

\begin{enumerate}
    \item
      ¿Cuáles de los siguientes enunciados podrían ser hipótesis científicas?
      \begin{enumerate}
          \item
            ¿La frecuencia fundamental aumenta con la edad?
          \item
            El sujeto X aprenderá la palabra \emph{casa} antes que la palabra
            \emph{examen}.
          \item
            La presencia de tonos en una lengua está influida por la humedad de
            la zona en que se habla.
          \item
            Las lenguas con menos hablantes posiblemente tienden a cambiar más
            rápidamente que las lenguas con muchos hablantes.
          \item
            La relación entre forma y significado es arbitraria.
          \item
            Los tweets de Donald Trump hacen caer el valor de la industria china.
          \item
            
      \end{enumerate}
    \item
      Donde sea posible, reformulá los enunciados en forma de hipótesis.
    \item
      Operacionalizá las variables involucradas y determiná de qué tipo es
      cada una.
    \item
      ¿Qué variables podrían estar influyendo y no están siendo tomadas en cuenta?
     \item
      Operacionalizá matemáticamente las hipótesis.
\end{enumerate}

\hypertarget{recoleccion-de-datos}{%
\section{Recolección de datos}\label{recoleccion-de-datos}}

La \textbf{recolección} de datos comienza solo después de haber
operacionalizado las variables y formulado las hipótesis. Por lo
general, no se estudia la población entera sino una muestra. Si queremos
que nuestros datos puedan generalizarse a la población, esta muestra
debe ser \textbf{representativa} (i.e., las distintas partes de la
población deben estar reflejadas en la muestra) y \textbf{balanceada}
(i.e., los tamaños de las partes de la muestra deben corresponderse con
las proporciones que presentan en la población). Esto muchas veces es un
ideal teórico porque con frecuencia no conocemos todas las partes y las
proporciones de la población. Una forma de obtener una muestra
representativa y balanceada es a partir de la randomización. Este es uno
de los principios más importantes de la recolección de datos.


\subsection{Tipos de estudios}

De acuerdo al grado en el que se pueden manipular las condiciones del estudio pueden distinguirse dos tipos: 

\begin{description}
    \item[\small{Experimento:}]
        Es un estudio en el cual las condiciones son asignadas de manera deliberada (y usualmente aleatoria) a individuos/sujetos/instancias temporales con el objetivo de ver si el efecto de estas condiciones en alguna característica particular. Las condiciones suelen llamarse \emph{tratamientos} y se corresponden con los niveles de la(s) variable(s) independiente(s). La característica observada, por su parte, se corresponde con la variable dependiente. Los individuos/objetos/instancias temporales son las \emph{unidades experimentales}.
    \item[\small{Estudio observacional:}]
        Es un estudio donde las condiciones no son asignadas ni controladas por la persona que investiga, sino simplemente observadas. Las condiciones son características inherentes de cada sujeto/objeto/instancia temporal. El interés sigue puesto en comparar los valores de la variable dependiente en función de los niveles de la variable independiente. Aquí podemos distinguir entre estudios prospectivos, donde van recolectándose observaciones a medida que van teniendo lugar sucesos de interés, y estudios retrospectivos, donde se recolectan los datos luego de ocurridos los sucesos.
\end{description}

En un estudio experimental es muy importante la manera en la que se seleccionen las observaciones que integran la muestra, dado que para poder generalizar los resultados a la población de interés es importante que las observaciones sean independientes entre sí y que los distintos grupos dentro de la población estén debidamente representados. En términos de la selección de las unidades experimentales, se pueden identificar por lo menos cuatro formas distintas de muestreo:

\begin{description}
    \item[\small{Muestreo aleatorio:}]
        Supone la posibilidad de elegir al azar las unidades experimentales sobre la totalidad de la población.
    \item[\small{Muestreo estratificado:}]
        Supone la división de la población en estratos, subgrupos que comparten semejanzas respecto de un determinado conjunto de variables. La muestra del experimento se realiza en base a la proporción de la población total que representa cada estrato.
    \item[\small{Muestreo por \emph{clusters}:}]
        Consiste en dividir a la población en grupos de manera aleatoria, seleccionar \textbf{n} grupos, y dentro de cada grupo muestrear aleatoriamente.
    \item[\small{Muestreo por conveniencia:}]
        Consiste en una selección basada en la accesibilidad relativa de distintos miembros de la población. Puede generar una cantidad importante de sesgo en los resultados, dado que puede estar asociado a la recolección de muestras no independientes entre sí, o a la subrrepresentación de grupos.
\end{description}




\hypertarget{ejercitacion-5}{%
\subsection{Ejercitación}\label{ejercitacion-5}}

\begin{enumerate}
    \item
        De los siguientes ejemplos, ¿cuáles son experimentos y cuáles estudios observacionales?
        \begin{enumerate}
            \item
                Se intentó identificar los efectos de distractores externos (ninguno, constante, variables) y de
                tipos de palabras (frutas, sustantivos, cualquier tipo) en la habilidad para memorizar palabras.
                Hay nueve (3 x 3) combinaciones de distracción y tipo de palabra. 36 sujetos fueron asignados al
                azar a las nueve combinaciones, con 4 sujetos por combinación. Para cada tipo de palabra, se preparó
                una lista con 30 términos. Cada sujeto estudió la lista que le fue asignada por 5 minutos. Luego de 
                una espera de 2 minutos, se le pidió a los sujetos que reciten la totalidad de palabras que
                recordaban y se midió la cantidad de palabras correctamente reportadas.
            \item
                Se intentó comparar la dificultad de lectura en dos revistas, \textbf{Personas} y \textbf{Personas
                Jóvenes}. Se seleccionaron cien oraciones al azar de la última edición de cada revista y, para cada
                una, se comparó el largo promedio de las oraciones en cantidad de letras.
            \item
                Se intentó determinar si el CI de les niñes está relacionado con haber sido amamantades o no. 
                Investigadores midieron el CI de una gran cantidad de estudiantes de primer grado en una ciudad 
                grande. Les investigadores además consultaron a las madres si les habían amamantado o no.
            \item
                Se intentó determinar si una reducción en el número de pop-ups mejoraba la experiencia de uso de un
                sitio web. Un grupo de 1000 suscriptores fueron seleccionados al azar. La mitad de ellos vieron
                aproximadamente la mitad de pop-ups al visitar el sitio, mientras que la otra mitad vio la cantidad
                usual. Tras dos semanas, se le solicitó a les suscriptores que completaran una encuesta de
                satisfacción.
            \end{enumerate}
    \item
        En una población compuesta por hablantes de hasta 25 años en un 43\%,
        hablantes de entre 26 y 50 años en un 29\%, hablantes de entre 51 y 75
        años en un 19\% y mayores de 76 en un 9\%, querés estudiar la
        influencia del trap en la lengua. Diseñá el estudio.
\end{enumerate}


\hypertarget{almacenamiento-de-datos}{%
\section{Almacenamiento de datos}\label{almacenamiento-de-datos}}

Una vez que recolectamos los datos (o mientras lo hacemos), es necesario
\textbf{almacenarlos} en un formato que nos permita anotarlos,
manipularlos y evaluarlos fácilmente. Para esto es recomendable el uso
de hojas de cálculo (e.g., LibreOffice Calc), bases de datos o R.

Un formato recomendado para el almacenamiento es el
\emph{case-by-variable} (véase Cuadro 4):

\begin{itemize}
    \item
      la primera fila contiene los nombres de las variables;
    \item
      las otras filas representan cada una un \emph{data point} (i.e., una
      observación determinada de la variable dependiente);
    \item
      la primera columna numera todos los \(n\) casos de 1 a \(n\) (esto
      permite identificar cada fila y restaurar el orden original);
    \item
      las otras columnas representan una sola variable o característica
      correspondiente a un determinado \emph{data point}; y
    \item
      la información faltante se anota usando \emph{un} símbolo (por
      ejemplo, ``NA'') y el mismo solo debe usarse para representar dicho
      significado.
\end{itemize}

\begin{longtable}[]{@{}llll@{}}
\caption{Una tabla que usa el formato \emph{case-by-variable} para codificar información sobre el
posicionamiento de partículas en inglés en función del largo del OD
medido en sílabas.}\tabularnewline
\toprule
\begin{minipage}[b]{0.08\columnwidth}\raggedright
Caso\strut
\end{minipage} & \begin{minipage}[b]{0.09\columnwidth}\raggedright
Orden\strut
\end{minipage} & \begin{minipage}[b]{0.09\columnwidth}\raggedright
Largo\strut
\end{minipage} & \begin{minipage}[b]{0.63\columnwidth}\raggedright
Oración\strut
\end{minipage}\tabularnewline
\midrule
\endfirsthead
\toprule
\begin{minipage}[b]{0.08\columnwidth}\raggedright
Caso\strut
\end{minipage} & \begin{minipage}[b]{0.09\columnwidth}\raggedright
Orden\strut
\end{minipage} & \begin{minipage}[b]{0.09\columnwidth}\raggedright
Largo\strut
\end{minipage} & \begin{minipage}[b]{0.63\columnwidth}\raggedright
Oración\strut
\end{minipage}\tabularnewline
\midrule
\endhead
\begin{minipage}[t]{0.08\columnwidth}\raggedright
1\strut
\end{minipage} & \begin{minipage}[t]{0.09\columnwidth}\raggedright
vpo\strut
\end{minipage} & \begin{minipage}[t]{0.09\columnwidth}\raggedright
2\strut
\end{minipage} & \begin{minipage}[t]{0.63\columnwidth}\raggedright
He turned on the lights.\strut
\end{minipage}\tabularnewline
\begin{minipage}[t]{0.08\columnwidth}\raggedright
2\strut
\end{minipage} & \begin{minipage}[t]{0.09\columnwidth}\raggedright
vpo\strut
\end{minipage} & \begin{minipage}[t]{0.09\columnwidth}\raggedright
2\strut
\end{minipage} & \begin{minipage}[t]{0.63\columnwidth}\raggedright
The police broke into the house.\strut
\end{minipage}\tabularnewline
\begin{minipage}[t]{0.08\columnwidth}\raggedright
3\strut
\end{minipage} & \begin{minipage}[t]{0.09\columnwidth}\raggedright
vop\strut
\end{minipage} & \begin{minipage}[t]{0.09\columnwidth}\raggedright
2\strut
\end{minipage} & \begin{minipage}[t]{0.63\columnwidth}\raggedright
Mary asked Susan out.\strut
\end{minipage}\tabularnewline
\begin{minipage}[t]{0.08\columnwidth}\raggedright
4\strut
\end{minipage} & \begin{minipage}[t]{0.09\columnwidth}\raggedright
vop\strut
\end{minipage} & \begin{minipage}[t]{0.09\columnwidth}\raggedright
2\strut
\end{minipage} & \begin{minipage}[t]{0.63\columnwidth}\raggedright
I had to hold my dog back because there was a cat in the park.\strut
\end{minipage}\tabularnewline
\begin{minipage}[t]{0.08\columnwidth}\raggedright
5\strut
\end{minipage} & \begin{minipage}[t]{0.09\columnwidth}\raggedright
vop\strut
\end{minipage} & \begin{minipage}[t]{0.09\columnwidth}\raggedright
2\strut
\end{minipage} & \begin{minipage}[t]{0.63\columnwidth}\raggedright
You can warm your feet up in front of the fireplace.\strut
\end{minipage}\tabularnewline
\begin{minipage}[t]{0.08\columnwidth}\raggedright
6\strut
\end{minipage} & \begin{minipage}[t]{0.09\columnwidth}\raggedright
vop\strut
\end{minipage} & \begin{minipage}[t]{0.09\columnwidth}\raggedright
3\strut
\end{minipage} & \begin{minipage}[t]{0.63\columnwidth}\raggedright
Our teacher finally broke the project down into three separate
parts.\strut
\end{minipage}\tabularnewline
\begin{minipage}[t]{0.08\columnwidth}\raggedright
7\strut
\end{minipage} & \begin{minipage}[t]{0.09\columnwidth}\raggedright
vpo\strut
\end{minipage} & \begin{minipage}[t]{0.09\columnwidth}\raggedright
3\strut
\end{minipage} & \begin{minipage}[t]{0.63\columnwidth}\raggedright
I'm looking for a red dress.\strut
\end{minipage}\tabularnewline
\begin{minipage}[t]{0.08\columnwidth}\raggedright
\ldots{}\strut
\end{minipage} & \begin{minipage}[t]{0.09\columnwidth}\raggedright
\ldots{}\strut
\end{minipage} & \begin{minipage}[t]{0.09\columnwidth}\raggedright
\ldots{}\strut
\end{minipage} & \begin{minipage}[t]{0.63\columnwidth}\raggedright
\ldots{}\strut
\end{minipage}\tabularnewline
\bottomrule
\end{longtable}

\hypertarget{ejercitacion-6}{%
\subsection{Ejercitación}\label{ejercitacion-6}}

\begin{enumerate}
    \item
      El dataset en el archivo datos\_lenguas.csv contiene información
      sobre el orden de sujeto, verbo y objeto de distintas lenguas y de la
      familia lingüística a la que pertenecen. Supongamos que querés investigar cómo este
      orden se ve influido por la familia lingüística. Creá un dataset que
      contenga esta información y se ajuste al formato
      \emph{case-by-variable}.
\end{enumerate}


\end{document}
