\documentclass[12pt,spanish,a4paper,]{article}
\usepackage{lmodern}
\usepackage{amssymb,amsmath}
\usepackage{ifxetex,ifluatex}
\usepackage{fixltx2e} % provides \textsubscript
\ifnum 0\ifxetex 1\fi\ifluatex 1\fi=0 % if pdftex
  \usepackage[T1]{fontenc}
  \usepackage[utf8]{inputenc}
\else % if luatex or xelatex
  \ifxetex
    \usepackage{mathspec}
  \else
    \usepackage{fontspec}
  \fi
  \defaultfontfeatures{Ligatures=TeX,Scale=MatchLowercase}
\fi
% use upquote if available, for straight quotes in verbatim environments
\IfFileExists{upquote.sty}{\usepackage{upquote}}{}
% use microtype if available
\IfFileExists{microtype.sty}{%
\usepackage{microtype}
\UseMicrotypeSet[protrusion]{basicmath} % disable protrusion for tt fonts
}{}
\usepackage[margin=1in]{geometry}
\usepackage{hyperref}
\PassOptionsToPackage{usenames,dvipsnames}{color} % color is loaded by hyperref
\hypersetup{unicode=true,
            pdftitle={Diseño y lógica de los estudios cuantitativos (versión Python)},
            pdfauthor={Santiago Gualchi},
            colorlinks=true,
            linkcolor=Maroon,
            citecolor=Blue,
            urlcolor=Blue,
            breaklinks=true}
\urlstyle{same}  % don't use monospace font for urls
\ifnum 0\ifxetex 1\fi\ifluatex 1\fi=0 % if pdftex
  \usepackage[main=spanish]{babel}
\else
  \usepackage{polyglossia}
  \setmainlanguage[]{spanish}
\fi
\usepackage{longtable,booktabs}
\usepackage{graphicx,grffile}
\makeatletter
\def\maxwidth{\ifdim\Gin@nat@width>\linewidth\linewidth\else\Gin@nat@width\fi}
\def\maxheight{\ifdim\Gin@nat@height>\textheight\textheight\else\Gin@nat@height\fi}
\makeatother
% Scale images if necessary, so that they will not overflow the page
% margins by default, and it is still possible to overwrite the defaults
% using explicit options in \includegraphics[width, height, ...]{}
\setkeys{Gin}{width=\maxwidth,height=\maxheight,keepaspectratio}
\usepackage[normalem]{ulem}
% avoid problems with \sout in headers with hyperref:
\pdfstringdefDisableCommands{\renewcommand{\sout}{}}
\IfFileExists{parskip.sty}{%
\usepackage{parskip}
}{% else
\setlength{\parindent}{0pt}
\setlength{\parskip}{6pt plus 2pt minus 1pt}
}
\setlength{\emergencystretch}{3em}  % prevent overfull lines
\providecommand{\tightlist}{%
  \setlength{\itemsep}{0pt}\setlength{\parskip}{0pt}}
\setcounter{secnumdepth}{5}
% Redefines (sub)paragraphs to behave more like sections
\ifx\paragraph\undefined\else
\let\oldparagraph\paragraph
\renewcommand{\paragraph}[1]{\oldparagraph{#1}\mbox{}}
\fi
\ifx\subparagraph\undefined\else
\let\oldsubparagraph\subparagraph
\renewcommand{\subparagraph}[1]{\oldsubparagraph{#1}\mbox{}}
\fi

%%% Use protect on footnotes to avoid problems with footnotes in titles
\let\rmarkdownfootnote\footnote%
\def\footnote{\protect\rmarkdownfootnote}

%%% Change title format to be more compact
\usepackage{titling}

% Create subtitle command for use in maketitle
\providecommand{\subtitle}[1]{
  \posttitle{
    \begin{center}\large#1\end{center}
    }
}

\setlength{\droptitle}{-2em}

  \title{Diseño y lógica de los estudios cuantitativos (versión Python)}
    \pretitle{\vspace{\droptitle}\centering\huge}
  \posttitle{\par}
    \author{Santiago Gualchi}
    \preauthor{\centering\large\emph}
  \postauthor{\par}
      \predate{\centering\large\emph}
  \postdate{\par}
    \date{21 de septiembre de 2019}


\begin{document}
\maketitle

\hypertarget{introduccion}{%
\section{Introducción}\label{introduccion}}

En la
\href{https://drive.google.com/file/d/1vRmq8vyfwkgANQznM2oPuzhorcoIaSkZ/view?usp=sharing}{clase
pasada}, realizamos una aproximación a la estadística y discutimos su
relevancia a la hora de estudiar el lenguaje. Señalamos la conocida
problemática de que, si bien los estudios cuantitativos están cada vez
más presentes en la investigación lingüística, en nuestro país (y, en
particular, en nuestra universidad), los planes de estudio excluyen la
estadística de la formación del lingüista. Esto supone un grave problema
para quienes nos formamos en el área en tanto que el desconocimiento de
los métodos usados actualmente en investigación amenaza con dejarnos
fuera del debate científico (no poder publicar, no poder leer).

A su vez, mencionamos que la estadística es la disciplina que se ocupa
de todas las etapas que involucran a los datos, incluyendo el diseño
previo a su recolección, su análisis e interpretación, y su
comunicación. En esta línea, introdujimos también una serie de
conceptos, entre ellos: población, muestra, variable, hipótesis, espacio
muestral y probabilidad.

Asimismo, establecimos la diferencia entre estadística descriptiva y
estadística inferencial. La primera se refiere al conjunto de técnicas
matemáticas que se limitan a describir las propiedades de la muestra
estudiada. La segunda alude a las pruebas que permiten generalizar las
observaciones sobre la muestra a la población relevante.

Finalmente, discutimos distintas áreas donde el uso de modelos
estadísticos puede ayudarnos a entender y a explicar mejor los fenómenos
que estudiamos. Entre los campos señalados incluimos (sin ser
exhaustivos) la psicolingüística y la neurolingüística, la lingüística
de corpus, la lingüística computacional, la tipología y la lingüística
histórica, y la teoría lingüística.

En esta reunión, vamos a avanzar sobre las líneas propuestas en el
encuentro anterior. Nos vamos a concentrar en las etapas de diseño de
una investigación para lo cual vamos a profundizar algunos de los
conceptos que ya introdujimos. Vamos a centrarnos en las hipótesis y
variables, y a estudiar sus propiedades y las repercusiones que traen
las distintas formas de operacionalizarlas. Vamos a explicar cómo
recolectar los datos de forma rigurosa y cuáles son los buenos hábitos
para su almacenamiento. Por último, vamos a introducir la forma de
proceder para la aceptación (o no) de nuestras hipótesis y por qué se
realiza de este modo.

\hypertarget{ejercitacion}{%
\subsection{Ejercitación}\label{ejercitacion}}

\begin{enumerate}
\def\labelenumi{\arabic{enumi}.}
\tightlist
\item
  Antes de avanzar, escribí definiciones para los siguientes conceptos:
  hipótesis, variable y operacionalización. No te preocupes si algunas
  de estas nociones te resultan muy nuevas. Está bien si las definís
  como te salga.
\item
  ¿Cómo caracterizarías los procesos de recolección y almacenamiento de
  datos? ¿Qué cuidados tendrías a la hora de llevarlos a cabo?
\item
  ¿Cómo pensás que podemos hacer para saber si nuestra hipótesis es
  correcta o incorrecta?
\end{enumerate}

\hypertarget{preparacion-del-directorio-de-trabajo}{%
\section{Preparación del directorio de
trabajo}\label{preparacion-del-directorio-de-trabajo}}

Cuando llevamos a cabo una investigación cuantitativa, trabajamos con
datos. Casi siempre estos datos son muy numerosos y necesitan ser
analizados con tests estadísticos que pueden involucrar un también muy
alto número de operaciones matemáticas. Por esta razón, hoy resulta
prácticamente imprescindible hacer uso de un software de análisis
estadístico (e.g., SPSS, SAS, Stata) o, mejor todavía, un lenguaje de
programación con buen soporte estadístico (e.g., R, Python, Julia). Para
mantener el espacio de trabajo organizado, facilitar la comprensión y
reproducibilidad y evitar errores prevenibles, es importante mantener
una buena estructuración de nuestro directorio. Existen distintas
prácticas sugeridas. A continuación voy a resumir las
\href{https://drivendata.github.io/cookiecutter-data-science/}{recomendaciones
desarrolladas por DrivenData} para la estructuración de proyectos de
data science en Python:

\begin{description}
\tightlist
\item[El archivo README]
Todos nos hemos encontrado con archivos que se llaman ``README'',
``README.md'', ``README.txt'', ``LÉAME'', ``LEEME.md'', ``LEER'',
``LEEMEEEE!!.txt'' y otras variantes cuando descargamos algunos
contenidos de internet. El archivo README es uno de los más importantes
de cualquier proyecto. Como su nombre sugiere es el primer lugar por el
que se accede al contenido. Un buen archivo README resume la información
esencial que el desarrollador considera que el usuario debe conocer al
hacer uso de su solución, y es el primer archivo que el usuario debe
consultar cuando accede a un proyecto. Los archivos README suelen estar
escritos en texto plano (sin ningún tipo de formato) o usando algún
lenguaje de marcado (por lo general, MarkDown). Es fundamental que todo
proyecto que encaremos cuente con un buen archivo README! Pueden ver un
ejemplo de
\href{https://github.com/sfbrigade/data-science-wg/blob/master/dswg_project_resources/Project-README-template.md}{una
buena plantilla de README} para proyectos en Data Science en el sitio de
Code for San Francisco.
\item[El archivo LICENSE]
Este es otro documento muy importante. Por un lado, resguarda tus
derechos de autor y te permite definir qué usos pueden hacer otros de tu
código. Pero también, permite a otros reutilizar los recursos que
desarrollaste. No incluir una licencia en tu proyecto puede suponer un
impedimento legal para que otros usen, modifiquen o reciclen tu código.
Con el fin de hacer una ciencia más abierta se anima cada vez más el uso
de licencias libres, que permiten al usuario el uso, el estudio, la
redistribución y la mejora del trabajo. Algunos ejemplos conocidos de
este tipos de licencias son
\href{https://www.gnu.org/licenses/gpl-3.0.txt}{GNU GPL},
\href{http://www.linfo.org/bsdlicense.html}{BSD} y
\href{https://opensource.org/licenses/MIT}{MIT}.
\item[El directorio data]
Todos los datos que vayamos a usar van a estar albergados en esta
carpeta. A su vez, dentro de esta carpeta, vamos a incluir una carpeta
\textbf{raw} en la que vamos a almacenar la versión original de nuestros
datos. Estos no deben ser \emph{nunca} modificados, sino que se deben
crear nuevas versiones y almacenarlas en otros directorios dentro
nuestra carpeta data.
\item[El directorio models]
Acá almacenamos nuestros modelos entrenados y serializados, sus
predicciones y sus resúmenes.
\item[El directorio notebooks]
En este directorio vamos a guardar las
\textbf{\href{https://jupyter.org/}{Jupyter Notebooks}} que escribamos.
Estas son una muy buena solución para combinar código ejecutable y
documentación. De este modo, podemos escribir reportes en forma
sencilla, pero también podemos aprovechar su versatilidad para correr
análisis exploratorios.
\item[El directorio src]
En esta carpeta va a estar contenido todo el código que escribamos para
llevar a cabo nuestra análisis, como el que se ocupa de descargar,
generar o limpiar los datos, definir y entrenar los modelos, o crear
visualizaciones. La única excepción es que vamos a incluir código que
usa el que está contenido en este directorio en las notebooks.
\end{description}

Esta es una simplificación de la propuesta, la estructura completa de
directorio incluye archivos y carpetas específicos para Python, además
de otros subdirectorios para usos especializados. Además, cabe señalar
que este diseño puede resultar útil para un gran número de proyectos,
pero no quiere decir que no existan casos en los que apartarse un poco
de estas recomendaciones pueda resultar productivo.

\hypertarget{ejercitacion-1}{%
\subsection{Ejercitación}\label{ejercitacion-1}}

\begin{enumerate}
\def\labelenumi{\arabic{enumi}.}
\tightlist
\item
  Armá un directorio para una investigación cuantitativa siguiendo los
  lineamientos propuestos por el equipo de DrivenData.
\item
  Escribí un README siguiendo las sugerencias de Code for San Francisco.
\item
  Investigá en internet qué lenguajes de marcado existen, cuál es la
  sintaxis básica de MarkDown y cuáles son sus ventajas y desventajas.
  Formateá el ``README'' que escribiste usando MarkDown y renombrá el
  archivo ``README.md''.
\item
  Incluí una licencia GNU LGPL. ¿En qué se diferencia con la licencia
  GNU GPL?
\item
  Contestá verdadero o falso y justificá:

  \begin{enumerate}
  \def\labelenumii{\alph{enumii}.}
  \tightlist
  \item
    No es necesario guardar una copia de nuestros datos originales.
  \item
    Las notebooks son especialmente útiles para la manipulación de datos
    y su modelado.
  \item
    Un proyecto sin una licencia es por defecto de dominio público.
  \item
    La mayor parte del código fuente debe ser almacenada en ``src''.
  \end{enumerate}
\end{enumerate}

\hypertarget{control-de-versiones}{%
\section{Control de versiones}\label{control-de-versiones}}

El control de versiones se refiere a la gestión de los cambios en un
archivo o un grupo de archivos. Es lo que hacemos cuando en una carpeta
tenemos ``trabajo.odt'', ``trabajo2.odt'', ``trabajo final.odt'',
``trabajo FINAL.odt'', ``trabajo FINALISIMO.odt'' y así \emph{ad
infinitum}. No obstante, existen mejores formas para hacer esto. Una de
las soluciones más usadas es \href{https://git-scm.com/}{Git}, que nos
permite restaurar versiones pero también trabajar colaborativamente,
entre otras muchas ventajas. No vamos a introducir su funcionamiento,
pero pueden acceder a la
\href{https://aladaspalabras.gitlab.io/hellogit/}{guía en español
escrita por Macarena Fernández}.

\hypertarget{ejercitacion-2}{%
\subsection{Ejercitación}\label{ejercitacion-2}}

\begin{enumerate}
\def\labelenumi{\arabic{enumi}.}
\tightlist
\item
  Comprobá si Git está instalado en tu computadora. Si no sabés cómo
  hacerlo, buscá en internet.
\item
  Si no tenés Git instalado, instalalo.
\item
  Investigá qué es un repositorio.
\item
  Creá un repositorio en tu computadora desde la terminal.
\item
  Abrí una cuenta en una plataforma de desarrollo colaborativo y creá un
  repositorio desde ahí.
\item
  Buscá en internet qué otros sistemas de control de versiones existen y
  qué servicios que usamos frecuentemente los usan.
\end{enumerate}

\hypertarget{scouting}{%
\section{\texorpdfstring{\emph{Scouting}}{Scouting}}\label{scouting}}

Al principio de una investigación se suelen llevar a cabo las siguientes
tareas:

\begin{itemize}
\tightlist
\item
  una primera caracterización del fenómeno;
\item
  estudio de la bibliografía relevante;
\item
  observación del fenómeno en escenarios naturales para posibilitar una
  primera generalización inductiva;
\item
  recolección de información adicional (e.g., de colegas, estudiantes,
  etc.);
\item
  razonamiento deductivo.
\end{itemize}

Si estudiamos el orden de palabras de los verbos frasales del inglés,
encontramos la siguiente alternancia:

\begin{enumerate}
\def\labelenumi{(\arabic{enumi})}
\item
  \begin{enumerate}
  \def\labelenumii{\alph{enumii}.}
  \item
    He picked up {[}\textsubscript{SN} the book{]}.

    Orden: \emph{VPO} (verbo - partícula - objeto)
  \item
    He picked {[}\textsubscript{SN} the book{]} up.

    Orden: \emph{VOP} (verbo - objeto- partícula)
  \end{enumerate}
\end{enumerate}

Al observar este fenómeno podemos encontrar un gran número de posibles
variables que podrían influir en la elección de una u otra forma. Las
\textbf{variables} son símbolos que pueden tomar, por lo menos, dos
estados o niveles diferentes (e.g., la edad de un grupo de estudiantes
de secundaria). En este sentido, se oponen a las \textbf{constantes},
que siempre presentan un mismo valor sin experimentar variación (e.g.,
la edad de un grupo de jóvenes de 12 años). Entre las variables que
pueden afectar al posicionamiento de la partícula en los verbos frasales
del inglés, las siguientes han sido propuestas en la bibliografía:

\begin{itemize}
\tightlist
\item
  Complejidad del OD (Fraser, 1966);
\item
  Largo del OD (Chen, 1986; Hawkins, 1994);
\item
  Presencia de un SP direccional (Chen, 1986);
\item
  Animacidad (Gries, 2003);
\item
  Concreción (Gries, 2003); y
\item
  Tipo del OD (Van Dongen, 1919), entre otras.
\end{itemize}

Esta información puede ser más fácilmente visualizada en formato
tabular, que permite reconocer qué variables han sido consideradas en
los distintos estudios y cuántas variables consideró cada estudio (véase
\protect\hyperlink{cuadro1}{Cuadro 1}). Otra tabla útil es la que
sintetiza los niveles de las variables y sus preferencias para uno u
otro orden. Como se ve en el \protect\hyperlink{cuadro2}{Cuadro 2}, el
orden \emph{VPO} sería usado con OODD cognitivamente más complejos (SSNN
complejos y largos con sustantivos léxicos que refieren a entidades
abstractas). \emph{VOP}, en cambio, es usado en los casos opuestos.

\begin{longtable}[]{@{}lccccc@{}}
\caption{\protect\hypertarget{cuadro1}{}{Resumen} de la bibliografía
sobre posicionamiento de partículas en inglés I.}\tabularnewline
\toprule
\begin{minipage}[b]{0.18\columnwidth}\raggedright
\strut
\end{minipage} & \begin{minipage}[b]{0.15\columnwidth}\centering
Van Dongen (1919)\strut
\end{minipage} & \begin{minipage}[b]{0.13\columnwidth}\centering
Fraser (1966)\strut
\end{minipage} & \begin{minipage}[b]{0.13\columnwidth}\centering
Chen (1986)\strut
\end{minipage} & \begin{minipage}[b]{0.13\columnwidth}\centering
Hawkins (1994)\strut
\end{minipage} & \begin{minipage}[b]{0.13\columnwidth}\centering
Gries (2003)\strut
\end{minipage}\tabularnewline
\midrule
\endfirsthead
\toprule
\begin{minipage}[b]{0.18\columnwidth}\raggedright
\strut
\end{minipage} & \begin{minipage}[b]{0.15\columnwidth}\centering
Van Dongen (1919)\strut
\end{minipage} & \begin{minipage}[b]{0.13\columnwidth}\centering
Fraser (1966)\strut
\end{minipage} & \begin{minipage}[b]{0.13\columnwidth}\centering
Chen (1986)\strut
\end{minipage} & \begin{minipage}[b]{0.13\columnwidth}\centering
Hawkins (1994)\strut
\end{minipage} & \begin{minipage}[b]{0.13\columnwidth}\centering
Gries (2003)\strut
\end{minipage}\tabularnewline
\midrule
\endhead
\begin{minipage}[t]{0.18\columnwidth}\raggedright
Complejidad\strut
\end{minipage} & \begin{minipage}[t]{0.15\columnwidth}\centering
\strut
\end{minipage} & \begin{minipage}[t]{0.13\columnwidth}\centering
\(\times\)\strut
\end{minipage} & \begin{minipage}[t]{0.13\columnwidth}\centering
\strut
\end{minipage} & \begin{minipage}[t]{0.13\columnwidth}\centering
\strut
\end{minipage} & \begin{minipage}[t]{0.13\columnwidth}\centering
\strut
\end{minipage}\tabularnewline
\begin{minipage}[t]{0.18\columnwidth}\raggedright
Largo\strut
\end{minipage} & \begin{minipage}[t]{0.15\columnwidth}\centering
\strut
\end{minipage} & \begin{minipage}[t]{0.13\columnwidth}\centering
\strut
\end{minipage} & \begin{minipage}[t]{0.13\columnwidth}\centering
\(\times\)\strut
\end{minipage} & \begin{minipage}[t]{0.13\columnwidth}\centering
\(\times\)\strut
\end{minipage} & \begin{minipage}[t]{0.13\columnwidth}\centering
\strut
\end{minipage}\tabularnewline
\begin{minipage}[t]{0.18\columnwidth}\raggedright
SP Direccional\strut
\end{minipage} & \begin{minipage}[t]{0.15\columnwidth}\centering
\strut
\end{minipage} & \begin{minipage}[t]{0.13\columnwidth}\centering
\strut
\end{minipage} & \begin{minipage}[t]{0.13\columnwidth}\centering
\(\times\)\strut
\end{minipage} & \begin{minipage}[t]{0.13\columnwidth}\centering
\strut
\end{minipage} & \begin{minipage}[t]{0.13\columnwidth}\centering
\strut
\end{minipage}\tabularnewline
\begin{minipage}[t]{0.18\columnwidth}\raggedright
Animacidad\strut
\end{minipage} & \begin{minipage}[t]{0.15\columnwidth}\centering
\strut
\end{minipage} & \begin{minipage}[t]{0.13\columnwidth}\centering
\strut
\end{minipage} & \begin{minipage}[t]{0.13\columnwidth}\centering
\strut
\end{minipage} & \begin{minipage}[t]{0.13\columnwidth}\centering
\strut
\end{minipage} & \begin{minipage}[t]{0.13\columnwidth}\centering
\(\times\)\strut
\end{minipage}\tabularnewline
\begin{minipage}[t]{0.18\columnwidth}\raggedright
Concreción\strut
\end{minipage} & \begin{minipage}[t]{0.15\columnwidth}\centering
\strut
\end{minipage} & \begin{minipage}[t]{0.13\columnwidth}\centering
\strut
\end{minipage} & \begin{minipage}[t]{0.13\columnwidth}\centering
\strut
\end{minipage} & \begin{minipage}[t]{0.13\columnwidth}\centering
\strut
\end{minipage} & \begin{minipage}[t]{0.13\columnwidth}\centering
\(\times\)\strut
\end{minipage}\tabularnewline
\begin{minipage}[t]{0.18\columnwidth}\raggedright
Tipo\strut
\end{minipage} & \begin{minipage}[t]{0.15\columnwidth}\centering
\(\times\)\strut
\end{minipage} & \begin{minipage}[t]{0.13\columnwidth}\centering
\strut
\end{minipage} & \begin{minipage}[t]{0.13\columnwidth}\centering
\strut
\end{minipage} & \begin{minipage}[t]{0.13\columnwidth}\centering
\strut
\end{minipage} & \begin{minipage}[t]{0.13\columnwidth}\centering
\strut
\end{minipage}\tabularnewline
\bottomrule
\end{longtable}

\begin{longtable}[]{@{}lll@{}}
\caption{\protect\hypertarget{cuadro2}{}{Resumen} de la bibliografía
sobre posicionamiento de partículas en inglés II.}\tabularnewline
\toprule
\begin{minipage}[b]{0.22\columnwidth}\raggedright
\strut
\end{minipage} & \begin{minipage}[b]{0.39\columnwidth}\raggedright
Nivel de variable para \emph{VPO}\strut
\end{minipage} & \begin{minipage}[b]{0.25\columnwidth}\raggedright
Nivel de variable para \emph{VOP}\strut
\end{minipage}\tabularnewline
\midrule
\endfirsthead
\toprule
\begin{minipage}[b]{0.22\columnwidth}\raggedright
\strut
\end{minipage} & \begin{minipage}[b]{0.39\columnwidth}\raggedright
Nivel de variable para \emph{VPO}\strut
\end{minipage} & \begin{minipage}[b]{0.25\columnwidth}\raggedright
Nivel de variable para \emph{VOP}\strut
\end{minipage}\tabularnewline
\midrule
\endhead
\begin{minipage}[t]{0.22\columnwidth}\raggedright
Complejidad\strut
\end{minipage} & \begin{minipage}[t]{0.39\columnwidth}\raggedright
modificado sintagmáticamente modificado por cláusula\strut
\end{minipage} & \begin{minipage}[t]{0.25\columnwidth}\raggedright
\strut
\end{minipage}\tabularnewline
\begin{minipage}[t]{0.22\columnwidth}\raggedright
Largo\strut
\end{minipage} & \begin{minipage}[t]{0.39\columnwidth}\raggedright
largo\strut
\end{minipage} & \begin{minipage}[t]{0.25\columnwidth}\raggedright
\strut
\end{minipage}\tabularnewline
\begin{minipage}[t]{0.22\columnwidth}\raggedright
SP Direccional\strut
\end{minipage} & \begin{minipage}[t]{0.39\columnwidth}\raggedright
ausente\strut
\end{minipage} & \begin{minipage}[t]{0.25\columnwidth}\raggedright
presente\strut
\end{minipage}\tabularnewline
\begin{minipage}[t]{0.22\columnwidth}\raggedright
Animacidad\strut
\end{minipage} & \begin{minipage}[t]{0.39\columnwidth}\raggedright
inanimado\strut
\end{minipage} & \begin{minipage}[t]{0.25\columnwidth}\raggedright
animado\strut
\end{minipage}\tabularnewline
\begin{minipage}[t]{0.22\columnwidth}\raggedright
Concreción\strut
\end{minipage} & \begin{minipage}[t]{0.39\columnwidth}\raggedright
abstracto\strut
\end{minipage} & \begin{minipage}[t]{0.25\columnwidth}\raggedright
concreto\strut
\end{minipage}\tabularnewline
\begin{minipage}[t]{0.22\columnwidth}\raggedright
Tipo\strut
\end{minipage} & \begin{minipage}[t]{0.39\columnwidth}\raggedright
\strut
\end{minipage} & \begin{minipage}[t]{0.25\columnwidth}\raggedright
pronominal\strut
\end{minipage}\tabularnewline
\bottomrule
\end{longtable}

\hypertarget{ejercitacion-3}{%
\subsection{Ejercitación}\label{ejercitacion-3}}

\begin{enumerate}
\def\labelenumi{\arabic{enumi}.}
\item
  Planteá un problema de investigación de tu interés.
\item
  ¿Cómo lo caracterizarías?
\item
  Completá la siguiente tabla con bibliografía que sea relevante para el
  problema que planteaste. En la columna observación, escribí por qué
  ese trabajo es particularmente útil para tu propuesta de
  investigación.

  \begin{longtable}[]{@{}llll@{}}
  \toprule
  \begin{minipage}[b]{0.22\columnwidth}\raggedright
  Autor\strut
  \end{minipage} & \begin{minipage}[b]{0.15\columnwidth}\raggedright
  Año\strut
  \end{minipage} & \begin{minipage}[b]{0.17\columnwidth}\raggedright
  Título\strut
  \end{minipage} & \begin{minipage}[b]{0.34\columnwidth}\raggedright
  Observación\strut
  \end{minipage}\tabularnewline
  \midrule
  \endhead
  \begin{minipage}[t]{0.22\columnwidth}\raggedright
  \strut
  \end{minipage} & \begin{minipage}[t]{0.15\columnwidth}\raggedright
  \strut
  \end{minipage} & \begin{minipage}[t]{0.17\columnwidth}\raggedright
  \strut
  \end{minipage} & \begin{minipage}[t]{0.34\columnwidth}\raggedright
  \strut
  \end{minipage}\tabularnewline
  \begin{minipage}[t]{0.22\columnwidth}\raggedright
  \strut
  \end{minipage} & \begin{minipage}[t]{0.15\columnwidth}\raggedright
  \strut
  \end{minipage} & \begin{minipage}[t]{0.17\columnwidth}\raggedright
  \strut
  \end{minipage} & \begin{minipage}[t]{0.34\columnwidth}\raggedright
  \strut
  \end{minipage}\tabularnewline
  \begin{minipage}[t]{0.22\columnwidth}\raggedright
  \strut
  \end{minipage} & \begin{minipage}[t]{0.15\columnwidth}\raggedright
  \strut
  \end{minipage} & \begin{minipage}[t]{0.17\columnwidth}\raggedright
  \strut
  \end{minipage} & \begin{minipage}[t]{0.34\columnwidth}\raggedright
  \strut
  \end{minipage}\tabularnewline
  \begin{minipage}[t]{0.22\columnwidth}\raggedright
  \strut
  \end{minipage} & \begin{minipage}[t]{0.15\columnwidth}\raggedright
  \strut
  \end{minipage} & \begin{minipage}[t]{0.17\columnwidth}\raggedright
  \strut
  \end{minipage} & \begin{minipage}[t]{0.34\columnwidth}\raggedright
  \strut
  \end{minipage}\tabularnewline
  \begin{minipage}[t]{0.22\columnwidth}\raggedright
  \strut
  \end{minipage} & \begin{minipage}[t]{0.15\columnwidth}\raggedright
  \strut
  \end{minipage} & \begin{minipage}[t]{0.17\columnwidth}\raggedright
  \strut
  \end{minipage} & \begin{minipage}[t]{0.34\columnwidth}\raggedright
  \strut
  \end{minipage}\tabularnewline
  \bottomrule
  \end{longtable}
\item
  Escribí una lista de las personas en tu lugar de estudio o trabajo que
  podrían sugerirte bibliografía o líneas de análisis para este problema
  en particular.
\item
  ¿Qué motores de búsqueda podés usar para encontrar investigaciones
  científicas? Usalos para encontrar bibliografía de utilidad para la
  investigación que propusiste.
\end{enumerate}

\hypertarget{hipotesis-y-operacionalizacion}{%
\section{Hipótesis y
operacionalización}\label{hipotesis-y-operacionalizacion}}

Una vez que tenemos una visión general del fenómeno que queremos
estudiar, es el momento de formular \textbf{hipótesis}.

\hypertarget{hipotesis-cientificas-en-forma-de-texto}{%
\subsection{Hipótesis científicas en forma de
texto}\label{hipotesis-cientificas-en-forma-de-texto}}

Las hipótesis pueden clasificarse en \textbf{hipótesis de tipo 1} y
\textbf{de tipo 2}. Las primeras se caracterizan por ser:

\begin{itemize}
\tightlist
\item
  enunciados generales ocupados de más de un evento singular;
\item
  enunciados con una estructura condicional (\emph{si\ldots,
  entonces\ldots{}}), o que, al menos, puede ser parafraseados como tal;
  y
\item
  potencialmente \textbf{falsables} (i.e., se pueden pensar eventos o
  situaciones que contradigan al enunciado) y \textbf{testeables} (i.e.,
  se pueden realizar pruebas que determinen la verdad o falsedad del
  enunciado).
\end{itemize}

Para el estudio del posicionamiento de partículas en inglés, podemos
pensar, por ejemplo, las siguientes hipótesis:

\begin{itemize}
\tightlist
\item
  \protect\hypertarget{h1}{}{si} el objeto directo de un verbo frasal
  transitivo es sintácticamente complejo, entonces los hablantes nativos
  producirán el orden de constituyentes \emph{VPO} más seguido que
  cuando el objeto directo es sintácticamente simple;
\item
  si el objeto directo de un verbo frasal transitivo es largo, entonces
  los hablantes nativos producirán el orden de constituyentes \emph{VPO}
  más seguido que cuando el objeto directo es corto; o
\item
  si una construcción de verbo-partícula es seguida por un SP
  direccional, entonces los hablantes nativos producirán el orden de
  constituyentes \emph{VOP} más seguido que cuando el SP direccional no
  está presente.
\end{itemize}

A su vez, las variables que consideremos pueden clasificarse según su
influencia:

\begin{description}
\tightlist
\item[variable independiente]
es la variable presente en la prótasis, y suele referirse a la causa de
los cambios/efectos. La variable independiente representa tratamientos o
condiciones que el investigador controla (directa o indirectamente) para
entender sus efectos sobre la variable dependiente.
\item[variable dependiente]
es la variable presente en la apódosis, cuyos valores, variación o
distribución se quieren explicar. La variable dependiente es la salida
que depende del tratamiento experimental o de lo que el investigador
cambia o manipula.
\item[\emph{confounder}]
es una variable que interactúa tanto con la variable independiente como
con la variable dependiente. Es importante identificar los confounders
para realizar mejores diseños experimentales y obtener resultados con
menos ruido.
\item[variable moderadora]
es una variable independiente secundaria que se selecciona para
determinar si afecta la relación entre la variable independiente y la
dependiente.
\end{description}

Por su parte, las hipótesis de tipo 2 contienen solo una variable
dependiente y ninguna variable independiente. En estos casos, la
hipótesis es un enunciado sobre los valores, variación o distribución de
la variable dependiente. Por ejemplo, ``\protect\hypertarget{h2}{}{los}
dos niveles de Orden no son igualmente frecuentes''. Así, podemos
definir una hipótesis como un enunciado acerca de la relación entre dos
o más variables, o acerca de una variable en un \textbf{contexto
muestral}, que se espera que aplique en contextos similares y/o para
objetos similares de la población.

Una vez que formulamos nuestra hipótesis, a la que vamos a llamar
\textbf{hipótesis} \textbf{alternativa} (H\textsubscript{1}), y antes de
recolectar datos, tenemos que definir las situaciones y estados de cosas
que van a falsar nuestra hipótesis. De este modo, definimos la
\textbf{hipótesis nula} (H\textsubscript{0}) como el opuesto lógico de
H\textsubscript{1} (predice la ausencia del efecto que enuncia
H\textsubscript{1}). La llamamos hipótesis nula porque se postula para
ser anulada con los datos de la investigación. Esto es importante,
porque la idea es que ambas hipótesis cubran todo el espacio de
resultados o \textbf{espacio muestral}, i.e., el conjunto de todos los
resultados teóricamente posibles. Por ejemplo:

\begin{itemize}
\tightlist
\item
  si el objeto directo de un verbo frasal transitivo es sintácticamente
  complejo, entonces los hablantes nativos \emph{no} producirán el orden
  de constituyentes VPO más seguido que cuando el objeto directo es
  sintácticamente simple (H\textsubscript{0} correspondiente a la
  \protect\hyperlink{h1}{primera hipótesis de tipo 1}); o
\item
  los dos niveles de Orden (VPO y VOP) \sout{\emph{no} no} son
  igualmente frecuentes (H\textsubscript{0} correspondiente a la
  \protect\hyperlink{h2}{hipótesis de tipo 2}).
\end{itemize}

Ahora bien, en algunas investigaciones es posible suponer que los
efectos o relaciones entre variables ocurran en una dirección
determinada (se desvíen de la H\textsubscript{0} hacia \emph{un} lado).
En estos casos, se dice que se establece una \textbf{hipótesis
direccional}. Por el contrario, las \textbf{hipótesis no direccionales}
solo predicen que existe un efecto o relación sin especificar la
dirección del efecto.

\hypertarget{operacionalizacion-de-variables}{%
\subsection{Operacionalización de
variables}\label{operacionalizacion-de-variables}}

Una vez que formulamos nuestra hipótesis, es importante encontrar un
modo de \textbf{operacionalizar} las variables. Esto supone decidir qué
será observado, contado, medido, etc. cuando investiguemos nuestras
hipótesis. Por ejemplo, si volvemos a las variables consideradas en la
bibliografía sobre el orden de palabras en los verbos frasales del
inglés, podemos operacionalizarlas como sigue:

\begin{itemize}
\tightlist
\item
  Complejidad: OD \emph{simple} (e.g., \emph{the book}), \emph{OD
  modificado} \emph{sintagmáticamente} (e.g., \emph{the book on the
  table}) u \emph{OD modificado por} \emph{cláusula} (e.g., \emph{the
  book I had bought in Europe});
\item
  Largo: el largo del OD medido en sílabas;
\item
  SP direccional: \emph{presencia} o \emph{ausencia} de un SP
  direccional (e.g., \emph{He picked the book up {[}\textsubscript{SP}
  from the table{]}});
\item
  Animacidad: \emph{animado} o \emph{inanimado};
\item
  Concreción: \emph{concreto} o \emph{abstracto}; y
\item
  Tipo del OD: \emph{pronominal} (e.g., \emph{He picked
  {[}\textsubscript{pron} him{]} up this morning}),
  \emph{semipronominal} (e.g., \emph{He picked {[}\textsubscript{semi}
  something{]} up from the floor}), \emph{léxico} (e.g., \emph{He picked
  {[}\textsubscript{léx} people{]} up this morning}) o \emph{nombre
  propio} (e.g., \emph{He picked {[}\textsubscript{prop} Peter{]} up
  this morning}).
\end{itemize}

Otro ejemplo, si queremos operacionalizar el conocimiento de una lengua
extranjera de una persona, podemos tomar en consideración:

\begin{itemize}
\tightlist
\item
  la complejidad de las oraciones que una persona puede formar en la
  lengua en cuestión;
\item
  el tiempo en segundos entre dos errores en la conversación;
\item
  el número de errores cada 100 palabras en un texto que la persona
  escriba en 90 minutos.
\end{itemize}

La operacionalización de variables involucra el uso de \textbf{niveles}
numéricos para representar estados de variables. Un número puede ser una
medida (e.g., 402 ms de tiempo de reacción), pero los niveles, i.e.,
estados discretos no numéricos, también pueden, teóricamente, ser
codificados usando números. Según los niveles de medida las variables
pueden clasificarse en:

\begin{description}
\tightlist
\item[variable nominal (o binaria)]
solo pueden tomar dos niveles diferentes y sus valores solo revelan que
los objetos con estos valores exhiben características diferentes (e.g.,
animacidad);
\item[variable categórica]
pueden tomar tres niveles diferentes o más y sus valores solo revelan
que los objetos con estos valores exhiben características diferentes
(e.g., aspecto);
\item[variable ordinal]
permiten distinguir categorías, pero también permiten rankear los
objetos de forma significativa (e.g., complejidad del OD); y
\item[variable cuantitativa (o de razón)]
además de distinguir categorías y rankear objetos, también permiten
comparar las diferencias y los ratios entre valores de forma
significativa (e.g., largo en sílabas).
\end{description}

\hypertarget{hipotesis-estadisticas-en-formato-estadisticomatematico}{%
\subsection{Hipótesis estadísticas en formato
estadístico/matemático}\label{hipotesis-estadisticas-en-formato-estadisticomatematico}}

Después de formular las hipótesis (H\textsubscript{0} y
H\textsubscript{1}) en forma de texto y definir cómo operacionalizar las
variables, es necesario formular dos \textbf{versiones estadísticas} de
las hipótesis. Esto significa expresar los resultados numéricos
esperados sobre la base de las hipótesis textuales. Dichos resultados
suelen involucrar una de las siguientes formas matemáticas:

\begin{itemize}
\tightlist
\item
  frecuencias;
\item
  promedios;
\item
  dispersiones;
\item
  correlaciones; o
\item
  distribuciones.
\end{itemize}

Este va a ser el formato que vamos a usar para evaluar la
\textbf{significancia} de nuestras hipótesis (véase más abajo), y su
definición va a depender directamente de cómo operacionalizamos las
variables. Por ejemplo, si nuestra hipótesis involucra la variable largo
del OD, su forma estadística no va a ser la misma si operacionalizamos
cuantitativamente como largo medido en número de sílabas o de forma
discreta como una variable categórica con niveles \emph{corto},
\emph{mediano} y \emph{largo.} En el primer caso, nuestras hipótesis
estadísticas van a poder referirse a la media del largo, mientras que
esto no es posible en el segundo caso. Tomando largo como una variable
categórica podríamos operacionalizar nuestras hipótesis, por ejemplo,
basándonos en conteos o frecuencias.

Retomemos la \protect\hyperlink{h1}{H\textsubscript{1} respecto de la
presencia/ausencia de un SP direccional}: si una construcción de
verbo-partícula es seguida por un SP direccional, entonces los hablantes
nativos producirán el orden de constituyentes VOP más seguido que cuando
el SP direccional no está presente. Si formulamos nuestras hipótesis
matemáticamente, obtenemos los siguientes resultados:

\[H_{1~\textrm{direccional}}: n_{\textrm{SSPP dir. en VPO}} < n_{\textrm{SSPP dir. en VOP}}\]
\[H_{1~\textrm{no direccional}}: n_{\textrm{SSPP dir. en VPO}} \neq n_{\textrm{SSPP dir. en VOP}}\]
\[H_{0}: n_{\textrm{SSPP dir. en VPO}} =n_{\textrm{SSPP dir. en VOP}}\]

\hypertarget{ejercitacion-4}{%
\subsection{Ejercitación}\label{ejercitacion-4}}

\begin{enumerate}
\def\labelenumi{\arabic{enumi}.}
\tightlist
\item
  ¿Cuáles de los siguientes enunciados podrían ser hipótesis?

  \begin{enumerate}
  \def\labelenumii{\alph{enumii}.}
  \tightlist
  \item
    La variación lingüística está dada por diferencias en las
    propiedades de las categorías funcionales.
  \item
    ¿La frecuencia fundamental aumenta con la edad?
  \item
    El sujeto LC aprenderá la palabra \emph{casa} antes que la palabra
    \emph{examen}.
  \item
    La presencia de tonos en una lengua está influida por las
    condiciones climáticas de la zona en que se habla.
  \item
    La presencia de tonos en una lengua está influida por la humedad de
    la zona en que se habla.
  \item
    Las lenguas con menos hablantes posiblemente tienden a cambiar más
    rápidamente que las lenguas con muchos hablantes.
  \item
    La relación entre forma y significado es arbitraria.
  \item
    Oumuamua era una nave interestelar que llevaba criaturas con un
    sistema simbólico doblemente articulado.
  \end{enumerate}
\item
  Reformulá los enunciados que no tienen las características de las
  hipótesis para que las tengan.
\item
  ¿De qué tipo es cada hipótesis?
\item
  Operacionalizalas matemáticamente.
\item
  Operacionalizá las variables involucradas y determiná de qué tipo es
  cada una.
\item
  ¿Qué \emph{confounders} y variables moderadoras podrían estar
  influyendo y no están siendo tomadas en cuenta?
\end{enumerate}

\hypertarget{recoleccion-de-datos}{%
\section{Recolección de datos}\label{recoleccion-de-datos}}

La \textbf{recolección} de datos comienza solo después de haber
operacionalizado las variables y formulado las hipótesis. Por lo
general, no se estudia la población entera sino una muestra. Si queremos
que nuestros datos puedan generalizarse a la población, esta muestra
debe ser \textbf{representativa} (i.e., las distintas partes de la
población deben estar reflejadas en la muestra) y \textbf{balanceada}
(i.e., los tamaños de las partes de la muestra deben corresponderse con
las proporciones que presentan en la población). Esto muchas veces es un
ideal teórico porque con frecuencia no conocemos todas las partes y las
proporciones de la población. Una forma de obtener una muestra más
representativa y balanceada es a partir de la randomización. Este es uno
de los principios más importantes de la recolección de datos.

\hypertarget{ejercitacion-5}{%
\subsection{Ejercitación}\label{ejercitacion-5}}

\begin{enumerate}
\def\labelenumi{\arabic{enumi}.}
\tightlist
\item
  Querés caracterizar los distintos usos del prefijo \emph{in-} en una
  población de 150.000 hablantes,

  \begin{enumerate}
  \def\labelenumii{\alph{enumii}.}
  \tightlist
  \item
    ¿cuántos participantes necesitarías para la recolección de datos?
  \item
    ¿qué subgrupos de tu población deberías representar?
  \item
    ¿creés que esos subgrupos presentarán un comportamiento
    diferenciado?
  \end{enumerate}
\item
  En una población compuesta por hablantes de hasta 25 años en un 43\%,
  hablantes de entre 26 y 50 años en un 29\%, hablantes de entre 51 y 75
  años en un 19\% y mayores de 76 en un 9\%, querés estudiar la
  influencia del trap en la lengua. ¿Cómo conformarías la muestra?
\end{enumerate}

\hypertarget{almacenamiento-de-datos}{%
\section{Almacenamiento de datos}\label{almacenamiento-de-datos}}

Una vez que recolectamos los datos (o mientras lo hacemos), es necesario
\textbf{almacenarlos} en un formato que nos permita anotarlos,
manipularlos y evaluarlos fácilmente. Para esto es recomendable el uso
de hojas de cálculo (e.g., LibreOffice Calc), bases de datos o R.

Un formato recomendado para el almacenamiento es el
\emph{case-by-variable} (véase \protect\hyperlink{cbv}{Cuadro 4}):

\begin{itemize}
\tightlist
\item
  la primera fila contiene los nombres de las variables;
\item
  las otras filas representan cada una un \emph{data point} (i.e., una
  observación determinada de la variable dependiente);
\item
  la primera columna numera todos los \(n\) casos de 1 a \(n\) (esto
  permite identificar cada fila y restaurar el orden original);
\item
  las otras columnas representan una sola variable o característica
  correspondiente a un determinado \emph{data point}; y
\item
  la información faltante se anota usando \emph{un} símbolo (por
  ejemplo, ``NA'') y el mismo solo debe usarse para representar dicho
  significado.
\end{itemize}

\begin{longtable}[]{@{}llll@{}}
\caption{\protect\hypertarget{cbv}{}{Una} tabla que usa el formato
\emph{case-by-variable} para codificar información sobre el
posicionamiento de partículas en inglés en función del largo del OD
medido en sílabas.}\tabularnewline
\toprule
\begin{minipage}[b]{0.08\columnwidth}\raggedright
Caso\strut
\end{minipage} & \begin{minipage}[b]{0.09\columnwidth}\raggedright
Orden\strut
\end{minipage} & \begin{minipage}[b]{0.09\columnwidth}\raggedright
Largo\strut
\end{minipage} & \begin{minipage}[b]{0.63\columnwidth}\raggedright
Oración\strut
\end{minipage}\tabularnewline
\midrule
\endfirsthead
\toprule
\begin{minipage}[b]{0.08\columnwidth}\raggedright
Caso\strut
\end{minipage} & \begin{minipage}[b]{0.09\columnwidth}\raggedright
Orden\strut
\end{minipage} & \begin{minipage}[b]{0.09\columnwidth}\raggedright
Largo\strut
\end{minipage} & \begin{minipage}[b]{0.63\columnwidth}\raggedright
Oración\strut
\end{minipage}\tabularnewline
\midrule
\endhead
\begin{minipage}[t]{0.08\columnwidth}\raggedright
1\strut
\end{minipage} & \begin{minipage}[t]{0.09\columnwidth}\raggedright
vpo\strut
\end{minipage} & \begin{minipage}[t]{0.09\columnwidth}\raggedright
2\strut
\end{minipage} & \begin{minipage}[t]{0.63\columnwidth}\raggedright
He turned on the lights.\strut
\end{minipage}\tabularnewline
\begin{minipage}[t]{0.08\columnwidth}\raggedright
2\strut
\end{minipage} & \begin{minipage}[t]{0.09\columnwidth}\raggedright
vpo\strut
\end{minipage} & \begin{minipage}[t]{0.09\columnwidth}\raggedright
2\strut
\end{minipage} & \begin{minipage}[t]{0.63\columnwidth}\raggedright
The police broke into the house.\strut
\end{minipage}\tabularnewline
\begin{minipage}[t]{0.08\columnwidth}\raggedright
3\strut
\end{minipage} & \begin{minipage}[t]{0.09\columnwidth}\raggedright
vop\strut
\end{minipage} & \begin{minipage}[t]{0.09\columnwidth}\raggedright
2\strut
\end{minipage} & \begin{minipage}[t]{0.63\columnwidth}\raggedright
Mary asked Susan out.\strut
\end{minipage}\tabularnewline
\begin{minipage}[t]{0.08\columnwidth}\raggedright
4\strut
\end{minipage} & \begin{minipage}[t]{0.09\columnwidth}\raggedright
vop\strut
\end{minipage} & \begin{minipage}[t]{0.09\columnwidth}\raggedright
2\strut
\end{minipage} & \begin{minipage}[t]{0.63\columnwidth}\raggedright
I had to hold my dog back because there was a cat in the park.\strut
\end{minipage}\tabularnewline
\begin{minipage}[t]{0.08\columnwidth}\raggedright
5\strut
\end{minipage} & \begin{minipage}[t]{0.09\columnwidth}\raggedright
vop\strut
\end{minipage} & \begin{minipage}[t]{0.09\columnwidth}\raggedright
2\strut
\end{minipage} & \begin{minipage}[t]{0.63\columnwidth}\raggedright
You can warm your feet up in front of the fireplace.\strut
\end{minipage}\tabularnewline
\begin{minipage}[t]{0.08\columnwidth}\raggedright
6\strut
\end{minipage} & \begin{minipage}[t]{0.09\columnwidth}\raggedright
vop\strut
\end{minipage} & \begin{minipage}[t]{0.09\columnwidth}\raggedright
3\strut
\end{minipage} & \begin{minipage}[t]{0.63\columnwidth}\raggedright
Our teacher finally broked the project down into three separate
parts.\strut
\end{minipage}\tabularnewline
\begin{minipage}[t]{0.08\columnwidth}\raggedright
7\strut
\end{minipage} & \begin{minipage}[t]{0.09\columnwidth}\raggedright
vpo\strut
\end{minipage} & \begin{minipage}[t]{0.09\columnwidth}\raggedright
3\strut
\end{minipage} & \begin{minipage}[t]{0.63\columnwidth}\raggedright
I'm looking for a red dress.\strut
\end{minipage}\tabularnewline
\begin{minipage}[t]{0.08\columnwidth}\raggedright
\ldots{}\strut
\end{minipage} & \begin{minipage}[t]{0.09\columnwidth}\raggedright
\ldots{}\strut
\end{minipage} & \begin{minipage}[t]{0.09\columnwidth}\raggedright
\ldots{}\strut
\end{minipage} & \begin{minipage}[t]{0.63\columnwidth}\raggedright
\ldots{}\strut
\end{minipage}\tabularnewline
\bottomrule
\end{longtable}

\hypertarget{ejercitacion-6}{%
\subsection{Ejercitación}\label{ejercitacion-6}}

\begin{enumerate}
\def\labelenumi{\arabic{enumi}.}
\tightlist
\item
  El dataset en \url{https://tinyurl.com/y4dl43ea} contiene información
  sobre el orden de sujeto, verbo y objeto de distintas lenguas y de la
  familia lingüística a la que pertenecen. Querés investigar cómo este
  orden se ve influido por la familia lingüística. Creá un dataset que
  contenga esta información y se ajuste al formato
  \emph{case-by-variable}.
\item
  Investigá sobre la importancia de usar formatos estándar para el
  almacenamiento de datos.
\item
  Cross-Linguistic Data Formats es una iniciativa que busca proponer
  estándares para la representación de datos translingüísticos para la
  investigación tipológica e histórica. ¿Qué principios guían su diseño?
  ¿Qué tipos de datos soporta actualmente?
\item
  ¿Qué bases de datos disponibles abiertamente en internet se ajustan a
  estándares definidos por esta iniciativa?
\end{enumerate}

\hypertarget{como-decidir}{%
\section{Cómo decidir}\label{como-decidir}}

Cuando ya almacenamos los datos, procedemos a evaluarlos con algún
\textbf{test estadístico}. Sin embargo, la forma de proceder que se
acostumbra en ciencias biológicas, psicología, ciencias sociales y
humanidades consiste no en probar que H\textsubscript{1} es correcta,
sino que la versión estadística de H\textsubscript{0} es improbable y,
por lo tanto, pueda ser rechazada. Ya que H\textsubscript{0} es la
contracara lógica de H\textsubscript{1}, esto apoya H\textsubscript{1}.

El \textbf{testeo de hipótesis} nos permite decidir si un
\textbf{efecto} observado se debe a relaciones reales entre las
variables o al azar\footnote{Realmete el testeo de hipótesis no siempre
  nos permite decidir. Algunos procedimientos (como vamos a ver a
  continuación) solo nos permiten calcular la probabilidad de que una
  observación se dé por azar.}. Dicho de otra forma, nos permite
justificar la preferencia por explicar un fenómeno por medio de una
interacción entre variables contra considerar que dicha interacción no
tiene influencia sobre el efecto observado. Tanto el estadista como las
editoriales, y los demás actores sociales quieren evitar perder tiempo y
plata analizando/interpretando/considerando conclusiones incorrectas.
Para sortear esto existen distintas técnicas. Uno de los procedimientos
que se utilizan (especialmente en psicología) es la \textbf{Prueba de
Significancia de la Hipótesis} \textbf{Nula} (NHST, por sus siglas en
inglés)\footnote{La NHST ha sido fuertemente discutida. Para un análisis
  crítico de este procedimiento, véase Cohen (1994) y Perezgonzalez
  (2015).}. En líneas muy generales, podemos definirla como sigue:

\begin{enumerate}
\def\labelenumi{\arabic{enumi}.}
\tightlist
\item
  definición del \textbf{nivel de significancia
  \emph{p}\textsubscript{crítico}} , que por lo general es 0,05;
\item
  análisis de los datos computando la probabilidad de un efecto \emph{e}
  (e.g., una distribución, una diferencia de medias, una correlación)
  usando las hipótesis estadísticas;
\item
  computación de la \textbf{probabilidad de error \emph{p}} (qué tan
  probable es encontrar \emph{e} o algo que se desvía aún más de
  H\textsubscript{0} cuando H\textsubscript{0} es verdadera); y
\item
  comparación de \emph{p}\textsubscript{crítico} y \emph{p} y decidir si
  \emph{p} \textless{} \emph{p}\textsubscript{crítico}; entonces podemos
  rechazar H\textsubscript{0} y aceptar H\textsubscript{1}.
\end{enumerate}

Si la probabilidad de error \emph{p} de un fenómeno es menor a
\emph{p}\textsubscript{crítico} podemos rechazar H\textsubscript{0} y
aceptar H\textsubscript{1}. Esto no significa que hayamos probado
H\textsubscript{1}, sino que la probabilidad de error \emph{p} es lo
suficientemente baja como para aceptar H\textsubscript{1}. La
probabilidad de error \emph{p} es conocida como \textbf{valor \emph{p}}.
El \protect\hyperlink{p}{Cuadro 5} recoge la semántica estándar de dicho
valor.

\hypertarget{p}{}

\begin{longtable}[]{@{}lll@{}}
\caption{Semántica de los valores de \emph{p}.}\tabularnewline
\toprule
\begin{minipage}[b]{0.30\columnwidth}\raggedright
Valor\strut
\end{minipage} & \begin{minipage}[b]{0.38\columnwidth}\raggedright
Significancia\strut
\end{minipage} & \begin{minipage}[b]{0.16\columnwidth}\raggedright
Indicación\strut
\end{minipage}\tabularnewline
\midrule
\endfirsthead
\toprule
\begin{minipage}[b]{0.30\columnwidth}\raggedright
Valor\strut
\end{minipage} & \begin{minipage}[b]{0.38\columnwidth}\raggedright
Significancia\strut
\end{minipage} & \begin{minipage}[b]{0.16\columnwidth}\raggedright
Indicación\strut
\end{minipage}\tabularnewline
\midrule
\endhead
\begin{minipage}[t]{0.30\columnwidth}\raggedright
\(p < 0.001\)\strut
\end{minipage} & \begin{minipage}[t]{0.38\columnwidth}\raggedright
altamente significativo\strut
\end{minipage} & \begin{minipage}[t]{0.16\columnwidth}\raggedright
***\strut
\end{minipage}\tabularnewline
\begin{minipage}[t]{0.30\columnwidth}\raggedright
\(0.001 \leq p < 0.01\)\strut
\end{minipage} & \begin{minipage}[t]{0.38\columnwidth}\raggedright
muy significativo\strut
\end{minipage} & \begin{minipage}[t]{0.16\columnwidth}\raggedright
**\strut
\end{minipage}\tabularnewline
\begin{minipage}[t]{0.30\columnwidth}\raggedright
\(0.01 \leq p < 0.05\)\strut
\end{minipage} & \begin{minipage}[t]{0.38\columnwidth}\raggedright
significativo\strut
\end{minipage} & \begin{minipage}[t]{0.16\columnwidth}\raggedright
*\strut
\end{minipage}\tabularnewline
\begin{minipage}[t]{0.30\columnwidth}\raggedright
\(0.05 \leq p < 0.1\)\strut
\end{minipage} & \begin{minipage}[t]{0.38\columnwidth}\raggedright
marginalmente significativo\strut
\end{minipage} & \begin{minipage}[t]{0.16\columnwidth}\raggedright
\emph{ms} o .\strut
\end{minipage}\tabularnewline
\begin{minipage}[t]{0.30\columnwidth}\raggedright
\(0.01 \leq p\)\strut
\end{minipage} & \begin{minipage}[t]{0.38\columnwidth}\raggedright
no significativo\strut
\end{minipage} & \begin{minipage}[t]{0.16\columnwidth}\raggedright
\strut
\end{minipage}\tabularnewline
\bottomrule
\end{longtable}

Al analizar la significancia del efecto, es correcto rechazar la
hipótesis nula cuando la hipótesis alternativa es verdadera, y aceptar
la hipótesis nula cuando esta es verdadera (véase
\protect\hyperlink{errores}{Cuadro 6}). Sin embargo, existen dos
combinaciones lógicas más. Un \textbf{error de tipo I} ocurre cuando la
hipótesis nula es verdadera y se rechaza. Por lo general, estos errores
deben ser evitados tanto como sea posible en la carrera del
investigador, y es lo que buscamos hacer cuando llevamos a cabo pruebas
de testeo de hipótesis. Asimismo, un \textbf{error de tipo II} ocurre
cuando la hipótesis alternativa es verdadera y se rechaza. Estos errores
suelen ser menos graves que los de tipo I, pero, de todos modos, debemos
reducir la posibilidad de que ocurran.

\hypertarget{errores}{}

\begin{longtable}[]{@{}lll@{}}
\caption{Errores de tipo I y II.}\tabularnewline
\toprule
\begin{minipage}[b]{0.19\columnwidth}\raggedright
\strut
\end{minipage} & \begin{minipage}[b]{0.25\columnwidth}\raggedright
H\textsubscript{0} es verdadera\strut
\end{minipage} & \begin{minipage}[b]{0.25\columnwidth}\raggedright
H\textsubscript{1} es verdadera\strut
\end{minipage}\tabularnewline
\midrule
\endfirsthead
\toprule
\begin{minipage}[b]{0.19\columnwidth}\raggedright
\strut
\end{minipage} & \begin{minipage}[b]{0.25\columnwidth}\raggedright
H\textsubscript{0} es verdadera\strut
\end{minipage} & \begin{minipage}[b]{0.25\columnwidth}\raggedright
H\textsubscript{1} es verdadera\strut
\end{minipage}\tabularnewline
\midrule
\endhead
\begin{minipage}[t]{0.19\columnwidth}\raggedright
Rechaza H\textsubscript{0}\strut
\end{minipage} & \begin{minipage}[t]{0.25\columnwidth}\raggedright
error de tipo I\strut
\end{minipage} & \begin{minipage}[t]{0.25\columnwidth}\raggedright
correcto\strut
\end{minipage}\tabularnewline
\begin{minipage}[t]{0.19\columnwidth}\raggedright
Acepta H\textsubscript{0}\strut
\end{minipage} & \begin{minipage}[t]{0.25\columnwidth}\raggedright
correcto\strut
\end{minipage} & \begin{minipage}[t]{0.25\columnwidth}\raggedright
error de tipo II\strut
\end{minipage}\tabularnewline
\bottomrule
\end{longtable}

\hypertarget{valores-p-de-una-cola-en-distribuciones-de-probabilidad-discretas}{%
\subsection{\texorpdfstring{Valores \emph{p} de una cola en
distribuciones de probabilidad
discretas}{Valores p de una cola en distribuciones de probabilidad discretas}}\label{valores-p-de-una-cola-en-distribuciones-de-probabilidad-discretas}}

Supongamos que queremos estudiar ambigüedad categorial. Para ello
encuestamos 3 sujetos acerca de si la palabra \emph{camino} es un verbo
o un nombre, asumiendo que los nombres son cognitivamente más
prominentes y que, por lo tanto, las respuestas serán mayores para esta
categoría:

\begin{description}
\tightlist
\item[H\textsubscript{0} textual]
ambas respuestas son igualmente frecuentes.
\item[H\textsubscript{1} textual]
\emph{nombre} es una respuesta más frecuente que \emph{verbo}.
\item[H\textsubscript{0} estadística]
los sujetos van a responder \emph{nombre} tantas veces como
\emph{verbo}: \(n_{\textrm{nombre}} = n_{\textrm{verbo}}\).
\item[H\textsubscript{1} estadística]
los sujetos van a responder \emph{nombre} más veces que \emph{verbo}:
\(n_{\textrm{nombre}} > n_{\textrm{verbo}}\).
\end{description}

¿Si los 3 sujetos responden que \emph{camino} es un nombre podemos
rechazar H\textsubscript{0} y aceptar H\textsubscript{1} asumiendo un
\emph{p}\textsubscript{crítico} = 0,05? El
\protect\hyperlink{prob}{Cuadro 7} sintetiza la probabilidad de cada
respuesta bajo el supuesto de que H\textsubscript{0} es verdadera. Las
columnas N y V representan \textbf{variables aleatorias}. Cada una de
ellas contabiliza la frecuencia de ocurrencia de un nivel para una
variable. De esta forma creamos dos nuevos espacios muestrales con una
cantidad reducida (y, por lo tanto, más manejable) de elementos (i.e.,
posibles resultados)\footnote{En este caso, solo es necesario definir
  una variable aleatoria, ya que ambas son complementarias :
  \(N = 3 - V\). Definimos dos variables aleatorias a modo ilustrativo.}.
La columna \emph{p}\textsubscript{resultado} representa la probabilidad
de que ocurra la combinación de respuestas de los sujetos
correspondiente. Dado que, bajo la H\textsubscript{0} asumimos que
\emph{nombre} y \emph{verbo} son respuestas igualmente probables, la
probabilidad de que un sujeto dado responda \emph{nombre} o responda
\emph{verbo} es la misma (\(P(nombre) + P(verbo) = 1\);
\(P(nombre) = P(verbo) = 0.5\)). La probabilidad de cada combinación
puede calcularse de dos formas. La primera es, sabiendo que la suma de
las probabilidades de las combinaciones debe sumar 1 y asumiendo que
cada combinación es igualmente probable, dividir 1 por el total de
elementos en el espacio muestral: \(1 \div 8 = 0.125\). Otra posibilidad
consiste en calcular el producto de las probabilidades de las 3
respuestas correspondientes a la combinación:
\(0.5 \times 0.5 \times 0.5 = 0.125\). Dado que, bajo
H\textsubscript{0}, la probabilidad de que los 3 sujetos respondan
nombre es de \(p = 0.125\) y que \(p > p_{\textrm{crítico}}\) no podemos
rechazar H\textsubscript{0}.

\hypertarget{prob}{}

\begin{longtable}[]{@{}llllll@{}}
\caption{Todos los resultados posibles de pedir a 3 sujetos que
clasifiquen \emph{camino} como un nombre o un verbo.}\tabularnewline
\toprule
\begin{minipage}[b]{0.13\columnwidth}\raggedright
Sujeto 1\strut
\end{minipage} & \begin{minipage}[b]{0.13\columnwidth}\raggedright
Sujeto 2\strut
\end{minipage} & \begin{minipage}[b]{0.13\columnwidth}\raggedright
Sujeto 3\strut
\end{minipage} & \begin{minipage}[b]{0.07\columnwidth}\raggedright
\emph{N}\strut
\end{minipage} & \begin{minipage}[b]{0.07\columnwidth}\raggedright
\emph{V}\strut
\end{minipage} & \begin{minipage}[b]{0.32\columnwidth}\raggedright
\(p_{\textrm{resultados}}\)\strut
\end{minipage}\tabularnewline
\midrule
\endfirsthead
\toprule
\begin{minipage}[b]{0.13\columnwidth}\raggedright
Sujeto 1\strut
\end{minipage} & \begin{minipage}[b]{0.13\columnwidth}\raggedright
Sujeto 2\strut
\end{minipage} & \begin{minipage}[b]{0.13\columnwidth}\raggedright
Sujeto 3\strut
\end{minipage} & \begin{minipage}[b]{0.07\columnwidth}\raggedright
\emph{N}\strut
\end{minipage} & \begin{minipage}[b]{0.07\columnwidth}\raggedright
\emph{V}\strut
\end{minipage} & \begin{minipage}[b]{0.32\columnwidth}\raggedright
\(p_{\textrm{resultados}}\)\strut
\end{minipage}\tabularnewline
\midrule
\endhead
\begin{minipage}[t]{0.13\columnwidth}\raggedright
nombre\strut
\end{minipage} & \begin{minipage}[t]{0.13\columnwidth}\raggedright
nombre\strut
\end{minipage} & \begin{minipage}[t]{0.13\columnwidth}\raggedright
nombre\strut
\end{minipage} & \begin{minipage}[t]{0.07\columnwidth}\raggedright
3\strut
\end{minipage} & \begin{minipage}[t]{0.07\columnwidth}\raggedright
0\strut
\end{minipage} & \begin{minipage}[t]{0.32\columnwidth}\raggedright
\(0.125\)\strut
\end{minipage}\tabularnewline
\begin{minipage}[t]{0.13\columnwidth}\raggedright
nombre\strut
\end{minipage} & \begin{minipage}[t]{0.13\columnwidth}\raggedright
nombre\strut
\end{minipage} & \begin{minipage}[t]{0.13\columnwidth}\raggedright
verbo\strut
\end{minipage} & \begin{minipage}[t]{0.07\columnwidth}\raggedright
2\strut
\end{minipage} & \begin{minipage}[t]{0.07\columnwidth}\raggedright
1\strut
\end{minipage} & \begin{minipage}[t]{0.32\columnwidth}\raggedright
\(0.125\)\strut
\end{minipage}\tabularnewline
\begin{minipage}[t]{0.13\columnwidth}\raggedright
nombre\strut
\end{minipage} & \begin{minipage}[t]{0.13\columnwidth}\raggedright
verbo\strut
\end{minipage} & \begin{minipage}[t]{0.13\columnwidth}\raggedright
nombre\strut
\end{minipage} & \begin{minipage}[t]{0.07\columnwidth}\raggedright
2\strut
\end{minipage} & \begin{minipage}[t]{0.07\columnwidth}\raggedright
1\strut
\end{minipage} & \begin{minipage}[t]{0.32\columnwidth}\raggedright
\(0.125\)\strut
\end{minipage}\tabularnewline
\begin{minipage}[t]{0.13\columnwidth}\raggedright
nombre\strut
\end{minipage} & \begin{minipage}[t]{0.13\columnwidth}\raggedright
verbo\strut
\end{minipage} & \begin{minipage}[t]{0.13\columnwidth}\raggedright
verbo\strut
\end{minipage} & \begin{minipage}[t]{0.07\columnwidth}\raggedright
1\strut
\end{minipage} & \begin{minipage}[t]{0.07\columnwidth}\raggedright
2\strut
\end{minipage} & \begin{minipage}[t]{0.32\columnwidth}\raggedright
\(0.125\)\strut
\end{minipage}\tabularnewline
\begin{minipage}[t]{0.13\columnwidth}\raggedright
verbo\strut
\end{minipage} & \begin{minipage}[t]{0.13\columnwidth}\raggedright
nombre\strut
\end{minipage} & \begin{minipage}[t]{0.13\columnwidth}\raggedright
nombre\strut
\end{minipage} & \begin{minipage}[t]{0.07\columnwidth}\raggedright
2\strut
\end{minipage} & \begin{minipage}[t]{0.07\columnwidth}\raggedright
1\strut
\end{minipage} & \begin{minipage}[t]{0.32\columnwidth}\raggedright
\(0.125\)\strut
\end{minipage}\tabularnewline
\begin{minipage}[t]{0.13\columnwidth}\raggedright
verbo\strut
\end{minipage} & \begin{minipage}[t]{0.13\columnwidth}\raggedright
nombre\strut
\end{minipage} & \begin{minipage}[t]{0.13\columnwidth}\raggedright
verbo\strut
\end{minipage} & \begin{minipage}[t]{0.07\columnwidth}\raggedright
1\strut
\end{minipage} & \begin{minipage}[t]{0.07\columnwidth}\raggedright
2\strut
\end{minipage} & \begin{minipage}[t]{0.32\columnwidth}\raggedright
\(0.125\)\strut
\end{minipage}\tabularnewline
\begin{minipage}[t]{0.13\columnwidth}\raggedright
verbo\strut
\end{minipage} & \begin{minipage}[t]{0.13\columnwidth}\raggedright
verbo\strut
\end{minipage} & \begin{minipage}[t]{0.13\columnwidth}\raggedright
nombre\strut
\end{minipage} & \begin{minipage}[t]{0.07\columnwidth}\raggedright
1\strut
\end{minipage} & \begin{minipage}[t]{0.07\columnwidth}\raggedright
2\strut
\end{minipage} & \begin{minipage}[t]{0.32\columnwidth}\raggedright
\(0.125\)\strut
\end{minipage}\tabularnewline
\begin{minipage}[t]{0.13\columnwidth}\raggedright
verbo\strut
\end{minipage} & \begin{minipage}[t]{0.13\columnwidth}\raggedright
verbo\strut
\end{minipage} & \begin{minipage}[t]{0.13\columnwidth}\raggedright
verbo\strut
\end{minipage} & \begin{minipage}[t]{0.07\columnwidth}\raggedright
0\strut
\end{minipage} & \begin{minipage}[t]{0.07\columnwidth}\raggedright
3\strut
\end{minipage} & \begin{minipage}[t]{0.32\columnwidth}\raggedright
\(0.125\)\strut
\end{minipage}\tabularnewline
\bottomrule
\end{longtable}

La distribución de probabilidad de cada resultado posible para 3 sujetos
queda recogida en el primer histograma de la Figura 1. Como se observa
en dicha figura, a medida que aumenta el número de sujetos la
distribución se asemeja cada vez más a la de la distribución gaussiana o
normal.

Ahora bien, si encuestamos a 100 personas, ¿podemos rechazar
H\textsubscript{0} si 59 responden \emph{nombre}? La respuesta es sí:
asumiendo H\textsubscript{0}, la probabilidad de que los sujetos
respondan \emph{nombre} 59 veces o más es de \(p = 0.044\) (véase Figura
3).

\hypertarget{valores-p-de-dos-colas-en-distribuciones-de-probabilidad-discretas}{%
\subsection{\texorpdfstring{Valores \emph{p} de dos colas en
distribuciones de probabilidad
discretas}{Valores p de dos colas en distribuciones de probabilidad discretas}}\label{valores-p-de-dos-colas-en-distribuciones-de-probabilidad-discretas}}

En la sección anterior, nuestra H\_1\_ era direccional: ``Si una palabra
puede ser analizada como nombre o como verbo, los sujetos responderán
\emph{nombre} más frecuentemente''. La prueba de significancia que
discutimos es una prueba de una cola, porque solo nos interesaba una
dirección en la que el resultado observado se desviaba del resultado
esperado. Si, en cambio, asumimos una H\textsubscript{1} no direccional
(por ejemplo, ``Si una palabra puede ser analizada como nombre o como
verbo, los hablantes responderán verbo y nombre con distinta
frecuencia''), tenemos que mirar hacia ambos lados del desvío:

\begin{description}
\tightlist
\item[H\textsubscript{0} estadística]
los sujetos van a responder \emph{nombre} tantas veces como
\emph{verbo}: \(n_{\textrm{nombre}} = n_{\textrm{verbo}}\).
\item[H\textsubscript{1} estadística]
los sujetos van a responder \emph{nombre} en un número distinto de veces
que \emph{verbo}: \(n_{\textrm{nombre}} \neq n_{\textrm{verbo}}\).
\end{description}

Ahora imaginemos que, una vez más, de 100 sujetos que responden si
\emph{camino} es un nombre o un verbo, 59 deciden que es un nombre. Ya
que nuestra hipótesis es no direccional y queremos calcular la
probabilidad de que ocurra dicho resultado u otro que se desvíe aún más
de H\textsubscript{0} cuando H\textsubscript{0} es verdadera, tenemos
que mirar hacia ambos lados de la distribución (que los sujetos
respondan que es nombre 59 o más veces y que los sujetos respondan que
es nombre 41 o menos veces). Así, obtenemos una frecuencia acumulada de
\(p = 0.089\) (véase Figura 4). Dado que este resultado es mayor al
\emph{p}\textsubscript{crítico} que definimos, no podemos rechazar
H\textsubscript{0}.

Cuando tenemos conocimiento previo sobre un fenómeno, podemos formular
una hipótesis direccional. Esto nos habilita a que el resultado
necesario para una conclusión significativa sea menos extremo que en el
caso de una hipótesis no direccional. En la mayoría de los casos, el
valor \emph{p} que obtenemos para un resultado con una hipótesis
direccional es la mitad del valor \emph{p} obtenido para una hipótesis
no direccional.

\hypertarget{ejercitacion-7}{%
\subsection{Ejercitación}\label{ejercitacion-7}}

\begin{enumerate}
\def\labelenumi{\arabic{enumi}.}
\item
  Completá el siguiente cuadro:

  \begin{longtable}[]{@{}lll@{}}
  \toprule
  \begin{minipage}[b]{0.40\columnwidth}\raggedright
  Valor\strut
  \end{minipage} & \begin{minipage}[b]{0.36\columnwidth}\raggedright
  Significancia\strut
  \end{minipage} & \begin{minipage}[b]{0.15\columnwidth}\raggedright
  Indicación\strut
  \end{minipage}\tabularnewline
  \midrule
  \endhead
  \begin{minipage}[t]{0.40\columnwidth}\raggedright
  \(p < 0.001\)\strut
  \end{minipage} & \begin{minipage}[t]{0.36\columnwidth}\raggedright
  \_\_\_\_\_\_\_\_\_\_\_\_\_\strut
  \end{minipage} & \begin{minipage}[t]{0.15\columnwidth}\raggedright
  ***\strut
  \end{minipage}\tabularnewline
  \begin{minipage}[t]{0.40\columnwidth}\raggedright
  \(0.001 \leq p < 0.01\)\strut
  \end{minipage} & \begin{minipage}[t]{0.36\columnwidth}\raggedright
  muy significativo\strut
  \end{minipage} & \begin{minipage}[t]{0.15\columnwidth}\raggedright
  \_\_\_\_\_\strut
  \end{minipage}\tabularnewline
  \begin{minipage}[t]{0.40\columnwidth}\raggedright
  \(0.01 \leq p <\)\_\_\_\strut
  \end{minipage} & \begin{minipage}[t]{0.36\columnwidth}\raggedright
  significativo\strut
  \end{minipage} & \begin{minipage}[t]{0.15\columnwidth}\raggedright
  *\strut
  \end{minipage}\tabularnewline
  \begin{minipage}[t]{0.40\columnwidth}\raggedright
  \(0.05\)\_\_\_ \(p\) \_\_\_ \(0.1\)\strut
  \end{minipage} & \begin{minipage}[t]{0.36\columnwidth}\raggedright
  marginalmente significativo\strut
  \end{minipage} & \begin{minipage}[t]{0.15\columnwidth}\raggedright
  \emph{ms} o .\strut
  \end{minipage}\tabularnewline
  \begin{minipage}[t]{0.40\columnwidth}\raggedright
  \_\_\_\_\_\_\_\_\_\_\_\_\strut
  \end{minipage} & \begin{minipage}[t]{0.36\columnwidth}\raggedright
  \_\_\_\_\_\_\_\_\_\_\_\_\_\strut
  \end{minipage} & \begin{minipage}[t]{0.15\columnwidth}\raggedright
  \_\_\_\_\_\strut
  \end{minipage}\tabularnewline
  \bottomrule
  \end{longtable}
\item
  Un investigador acepta su hipótesis alternativa (que la sonorización
  de oclusivas sordas en español está influida por el signo zodiacal y
  el ascendente) y rechaza la nula. No obstante, el corpus de datos
  presentaba un sesgo importante y, en realidad, tal efecto no existe.
  ¿Qué tipo de error cometió este investigador?
\item
  Se quiere estudiar, el uso de tiempos compuestos en español
  rioplatense a partir de respuestas a preguntas. Se obtuvieron los
  siguientes resultados:

  \begin{longtable}[]{@{}llll@{}}
  \toprule
  \begin{minipage}[b]{0.06\columnwidth}\raggedright
  Caso\strut
  \end{minipage} & \begin{minipage}[b]{0.09\columnwidth}\raggedright
  Sujeto\strut
  \end{minipage} & \begin{minipage}[b]{0.12\columnwidth}\raggedright
  Respuesta\strut
  \end{minipage} & \begin{minipage}[b]{0.09\columnwidth}\raggedright
  Tiempo\strut
  \end{minipage}\tabularnewline
  \midrule
  \endhead
  \begin{minipage}[t]{0.06\columnwidth}\raggedright
  1\strut
  \end{minipage} & \begin{minipage}[t]{0.09\columnwidth}\raggedright
  1\strut
  \end{minipage} & \begin{minipage}[t]{0.12\columnwidth}\raggedright
  1\strut
  \end{minipage} & \begin{minipage}[t]{0.09\columnwidth}\raggedright
  S\strut
  \end{minipage}\tabularnewline
  \begin{minipage}[t]{0.06\columnwidth}\raggedright
  2\strut
  \end{minipage} & \begin{minipage}[t]{0.09\columnwidth}\raggedright
  1\strut
  \end{minipage} & \begin{minipage}[t]{0.12\columnwidth}\raggedright
  2\strut
  \end{minipage} & \begin{minipage}[t]{0.09\columnwidth}\raggedright
  S\strut
  \end{minipage}\tabularnewline
  \begin{minipage}[t]{0.06\columnwidth}\raggedright
  3\strut
  \end{minipage} & \begin{minipage}[t]{0.09\columnwidth}\raggedright
  2\strut
  \end{minipage} & \begin{minipage}[t]{0.12\columnwidth}\raggedright
  3\strut
  \end{minipage} & \begin{minipage}[t]{0.09\columnwidth}\raggedright
  S\strut
  \end{minipage}\tabularnewline
  \begin{minipage}[t]{0.06\columnwidth}\raggedright
  4\strut
  \end{minipage} & \begin{minipage}[t]{0.09\columnwidth}\raggedright
  2\strut
  \end{minipage} & \begin{minipage}[t]{0.12\columnwidth}\raggedright
  4\strut
  \end{minipage} & \begin{minipage}[t]{0.09\columnwidth}\raggedright
  S\strut
  \end{minipage}\tabularnewline
  \begin{minipage}[t]{0.06\columnwidth}\raggedright
  5\strut
  \end{minipage} & \begin{minipage}[t]{0.09\columnwidth}\raggedright
  3\strut
  \end{minipage} & \begin{minipage}[t]{0.12\columnwidth}\raggedright
  5\strut
  \end{minipage} & \begin{minipage}[t]{0.09\columnwidth}\raggedright
  C\strut
  \end{minipage}\tabularnewline
  \begin{minipage}[t]{0.06\columnwidth}\raggedright
  6\strut
  \end{minipage} & \begin{minipage}[t]{0.09\columnwidth}\raggedright
  3\strut
  \end{minipage} & \begin{minipage}[t]{0.12\columnwidth}\raggedright
  6\strut
  \end{minipage} & \begin{minipage}[t]{0.09\columnwidth}\raggedright
  C\strut
  \end{minipage}\tabularnewline
  \begin{minipage}[t]{0.06\columnwidth}\raggedright
  7\strut
  \end{minipage} & \begin{minipage}[t]{0.09\columnwidth}\raggedright
  4\strut
  \end{minipage} & \begin{minipage}[t]{0.12\columnwidth}\raggedright
  7\strut
  \end{minipage} & \begin{minipage}[t]{0.09\columnwidth}\raggedright
  S\strut
  \end{minipage}\tabularnewline
  \begin{minipage}[t]{0.06\columnwidth}\raggedright
  8\strut
  \end{minipage} & \begin{minipage}[t]{0.09\columnwidth}\raggedright
  4\strut
  \end{minipage} & \begin{minipage}[t]{0.12\columnwidth}\raggedright
  8\strut
  \end{minipage} & \begin{minipage}[t]{0.09\columnwidth}\raggedright
  C\strut
  \end{minipage}\tabularnewline
  \bottomrule
  \end{longtable}

  ¿Cuál es la probabilidad de obtener estos datos bajo la hipótesis nula
  de que no hay diferencias de frecuencia entre el uso de tiempos
  simples y tiempos compuestos u otros que se desvíen todavía más de
  esta hipótesis? ¿La probabilidad obtenida es significativa?
\item
  Graficá la distribución de probabilidades de la variable aleatoria
  tiempo simple e indicá en el gráfico dónde se ubican los datos
  obtenidos.
\end{enumerate}

\hypertarget{el-reporte}{%
\section{El reporte}\label{el-reporte}}

Uno de los pilares sobre los que se basa la ciencia es la replicabilidad
de los resultados de una investigación. Para asegurar que nuestro
estudio presente esta característica, tenemos que ser tan detallados
como sea posible.

El \textbf{reporte} de una investigación cuantitativa consiste, por lo
general, de cuatro partes: \textbf{introducción}, \textbf{métodos},
\textbf{resultados}, y \textbf{discusión}. Si se discute más de un caso
de estudio, en el informe, cada caso suele requerir sus propias
secciones de métodos, resultados y discusión, seguido de una discusión
general. Entre la información a presentar, tenemos que incluir la
población estudiada, las hipótesis consideradas y las variables (sin
dejar de lado su operacionalización), la confección de la muestra (y las
consideraciones que tuvimos en cuenta para que la misma sea
representativa y balanceada), la forma en que se almacenaron los datos y
los distintos pasos del testeo de hipótesis. Por último, es importante
no olvidar que el objetivo final de toda investigación es la
comunicación. En este sentido, tenemos que tener en cuenta el poder
ilustrativo que tienen los gráficos y las tablas para transmitir
información y, por supuesto, tratar de ser tan claros en las
explicaciones como sea posible!

\hypertarget{ejercitacion-8}{%
\subsection{Ejercitación}\label{ejercitacion-8}}

\begin{enumerate}
\def\labelenumi{\arabic{enumi}.}
\tightlist
\item
  Buscá un artículo sobre una investigación experimental que organice su
  comunicación siguiendo este formato. ¿Qué información incluye cada
  sección?
\end{enumerate}

\hypertarget{conclusion}{%
\section{Conclusión}\label{conclusion}}

En esta clase, discutimos cómo llevar a cabo una investigación
cuantitativa. Para ello, analizamos las distintas etapas que involucra
llevar a cabo un estudio siguiendo esta metodología, desde el
planeamiento hasta la redacción del reporte. Es crucial para obtener
resultados rigurosos atravesar cada una de las etapas de forma
responsable y detallada.


\end{document}
